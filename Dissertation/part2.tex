\chapter{Методика разработки и исследования динамики систем управления бортовыми оптико-электронными приборами с применением информационных технологий} \label{ch:ch2}

В настоящее время наиболее распространены в народном хозяйстве и военной технике комплексированные оптико - электронные системы (КОЭП) визуализации, включающие в себя каналы наблюдения и зондирования в широком спектре волн оптического диапазона \cite[]{Tarasov},\cite[]{Belyakov},\cite[]{Karpov},\cite[]{Torshina}. Появление на рынке матричных фотоприемников расширило возможности КОЭП и К(комплексов). Несмотря на интенсивное развитие   теории и методов расчета и совершенствование бортовых автоматических КОЭП и К  возникают ряд вопросов, влияющих на качество получаемой оптической информации, в частности – это вопросы динамики и качества управления и увязки их с оптическими характеристиками каналов визуализации \cite[]{Belyakov},\cite[]{Karpov}, \cite[]{Baloev16}, \cite[]{Karpov17}.

Рассматриваются алгоритмы методики разработки и исследования динамики систем виброзащиты и управления бортовыми оптико- электронными приборами в виде десяти последовательных интерактивных замкнутых процедур от анализа технического задания до испытаний на борту.

Оптико-электронные приборы (ОЭП) авиационного, морского, наземного и космического базирования нашли широкое применение при выполнении задач наблюдения и охраны, в том числе при решении народнохозяйственных задач и задач обороны и безопасности.

В настоящее время широко рекламируются комплексированные ОЭП визуализации, включающие в себя каналы наблюдения и зондирования в широком спектре волн оптического диапазона, для применения в народном хозяйстве и военной технике \cite[]{Tarasov},\cite[]{Belyakov}, \cite[]{Torshina}, \cite[]{Ivanov18}. Появление на рынке матричных фотоприемников расширило возможности ОЭП и комплексов. Несмотря на интенсивное развитие   теории и методов расчета и совершенствование бортовых автоматических ОЭП возникает ряд вопросов, влияющих на качество получаемой оптической информации, в частности – это вопросы динамики и качества управления и увязки их с характеристиками каналов визуализации \cite[]{Tarasov},\cite[]{Belyakov}, \cite[]{Baloev16}, \cite[]{Karpov17}, \cite[]{Gerasin19}. Ниже в развитие работы \cite[]{Tarasov} рассматривается методика разработки и исследования систем автоматического управления (САУ) и виброзащиты (СВ) ОЭП с использованием замкнутых процедур исследования от разработки математических моделей до испытаний на борту носителя.


\section{Методика разработки и исследования динамики} \label{sec:ch2/sec1-}


Разработка СВ и САУ ОЭП начинается с выбора приемлемого варианта. Для этого решается много критерийная задача оптимизации с учетом ряда противоречивых технико-экономических (Т-Э) требований: точность САУ -  $\alpha_{1}$, полосы пропускания САУ и СВ - $\alpha_{21}$, $\alpha_{22}$; качество изображения -  $\alpha_{3}$, время экспозиции -  $\alpha_{4}$ , потребляемая энергия -  $\alpha_{5}$, надежность -  $\alpha_{6}$, масса объекта управления (ОУ) -  $\alpha_{7}$, стоимость -  $\alpha_{8}$, конкурентоспособность - $\alpha_{9}$, характеристики объектива -$\alpha_{10}$ и приемника излучения - $\alpha_{11}$ и т.п., которые определяются из технического задания (ТЗ) на ОЭП методом экспертных оценок специалистов в этих областях науки и техники с учетом предварительных расчетов и исследований. Критерий выбора приемлемого (i) варианта определяется по формуле:

\begin{equation}
\label{eq:p2:1}
\begin{alignedat}{2}
k_i=min\sum_{j=1}^n{\gamma _{ji}\alpha _{ji}}
\end{alignedat}
\end{equation}

\section{Оценка допуска на точность стабилизации изображения} \label{sec:ch2/sec2}

\section{Разработка математической модели} \label{sec:ch2/sec3}

\section{Верификация параметров} \label{sec:ch2/sec4}

\section{Оценки декомпозируемости каналов управления, основанные на анализе устойчивости и качества регулирования каналов управления с учетом перекрестных связей в частотной области} \label{sec:ch2/sec5}

\section{Синтез регуляторов частотным методом} \label{sec:ch2/sec6}

\section{Разработка КИМ и исследование динамики пространственной модели} \label{sec:ch2/sec7}

\section{Разработка алгоритмов управления БОЭП} \label{sec:ch2/sec8}

\section{Разработка компьютерной имитационной модели} \label{sec:ch2/sec9}

\section{Макетные испытания} \label{sec:ch2/sec10}

\section{Выводы по главе} \label{sec:ch2/sec11}



ll
