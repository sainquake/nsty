\chapter{Методика разработки и исследования динамики систем управления бортовыми оптико-электронными приборами с применением информационных технологий} \label{ch:ch2}

В настоящее время наиболее распространены в народном хозяйстве и военной технике комплексированные оптико - электронные системы (КОЭП) визуализации, включающие в себя каналы наблюдения и зондирования в широком спектре волн оптического диапазона \cite[]{Tarasov},\cite[]{Belyakov},\cite[]{Karpov},\cite[]{Torshina}. Появление на рынке матричных фотоприемников расширило возможности КОЭП и К(комплексов). Несмотря на интенсивное развитие   теории и методов расчета и совершенствование бортовых автоматических КОЭП и К  возникают ряд вопросов, влияющих на качество получаемой оптической информации, в частности – это вопросы динамики и качества управления и увязки их с оптическими характеристиками каналов визуализации \cite[]{Belyakov},\cite[]{Karpov}, \cite[]{Baloev16}, \cite[]{Karpov17}.

Рассматриваются алгоритмы методики разработки и исследования динамики систем виброзащиты и управления бортовыми оптико- электронными приборами в виде десяти последовательных интерактивных замкнутых процедур от анализа технического задания до испытаний на борту.

Оптико-электронные приборы (\hyperref[acroEOS]{ОЭП}) авиационного, морского, наземного и космического базирования нашли широкое применение при выполнении задач наблюдения и охраны, в том числе при решении народнохозяйственных задач и задач обороны и безопасности.

В настоящее время широко рекламируются комплексированные \hyperref[acroEOS]{ОЭП} визуализации, включающие в себя каналы наблюдения и зондирования в широком спектре волн оптического диапазона, для применения в народном хозяйстве и военной технике \cite[]{Tarasov},\cite[]{Belyakov}, \cite[]{Torshina}, \cite[]{Ivanov18}. Появление на рынке матричных фотоприемников расширило возможности \hyperref[acroEOS]{ОЭП} и комплексов. Несмотря на интенсивное развитие   теории и методов расчета и совершенствование бортовых автоматических \hyperref[acroEOS]{ОЭП} возникает ряд вопросов, влияющих на качество получаемой оптической информации, в частности – это вопросы динамики и качества управления и увязки их с характеристиками каналов визуализации \cite[]{Tarasov},\cite[]{Belyakov}, \cite[]{Baloev16}, \cite[]{Karpov17}, \cite[]{Gerasin19}. Ниже в развитие работы \cite[]{Tarasov} рассматривается методика разработки и исследования систем автоматического управления (\hyperref[acroSAU]{САУ}) и виброзащиты (СВ) \hyperref[acroEOS]{ОЭП} с использованием замкнутых процедур исследования от разработки математических моделей до испытаний на борту носителя.


\section{Методика разработки и исследования динамики} \label{sec:ch2/sec1-}


Разработка СВ и \hyperref[acroSAU]{САУ} \hyperref[acroEOS]{ОЭП} начинается с выбора приемлемого варианта. Для этого решается много критерийная задача оптимизации с учетом ряда противоречивых технико-экономических (Т-Э) требований: точность \hyperref[acroSAU]{САУ} -  $\alpha_{1}$, полосы пропускания \hyperref[acroSAU]{САУ} и СВ - $\alpha_{21}$, $\alpha_{22}$; качество изображения -  $\alpha_{3}$, время экспозиции -  $\alpha_{4}$ , потребляемая энергия -  $\alpha_{5}$, надежность -  $\alpha_{6}$, масса объекта управления (ОУ) -  $\alpha_{7}$, стоимость -  $\alpha_{8}$, конкурентоспособность - $\alpha_{9}$, характеристики объектива -$\alpha_{10}$ и приемника излучения - $\alpha_{11}$ и т.п., которые определяются из технического задания (ТЗ) на \hyperref[acroEOS]{ОЭП} методом экспертных оценок специалистов в этих областях науки и техники с учетом предварительных расчетов и исследований. Критерий выбора приемлемого (i) варианта определяется по формуле:

\begin{equation}
\label{eq:p2:1}
\begin{alignedat}{2}
k_i=min\sum_{j=1}^n{\gamma _{ji}\alpha _{ji}}
\end{alignedat}
\end{equation}

где \textit{i} – число вариантов, \textit{n} – число Т-Э параметров, $\gamma_{ij}$ – весовые коэффициенты, $\alpha_{ji}$ – Т-Э параметры. Для выбранных приемлемых вариантов  СВ и \hyperref[acroSAU]{САУ} (одного или двух) в соответствии с методикой изложенной в главе \ref{ch:ch4} проводится исследование их динамики. За критерий качества \hyperref[acroSAU]{САУ} обычно принимают совокупность динамических характеристик каналов управления, удовлетворяющих условиям:

\begin{equation}
\label{eq:p2:2}
\begin{alignedat}{2}
\varDelta \alpha _k\leqslant \varDelta \alpha _{k}^{\textit{доп}},
\\
\,\,\,\,\varDelta \dot{\alpha}_k\leqslant \varDelta \dot{\alpha}_{k}^{\textit{доп}},
\\
\,\,\,\,\,\,M\leqslant \text{1.05..1.25,}
\\
\,\,\,\,\left| \left. \varDelta \varphi \right| \right. \geqslant \left( 45-60 \right) ^0,
\\
\,\,\,\,\left| \left. \varDelta L \right|\geqslant \,\,\textit{6дб}, \right. 
\end{alignedat}
\end{equation}

где  $\varDelta \alpha^{\textit{доп}}_{k}$, 
$\varDelta \dot{\alpha}^{\textit{доп}}_{k}$, 
$\varDelta \alpha _k$, 
$\varDelta \dot{\alpha}_k$ 
- допустимые установившиеся значения динамических погрешностей \hyperref[acroSAU]{САУ} и СВ по углу  и угловой скорости  при действии возмущений, полученных в условиях, близких к реальной эксплуатации \hyperref[acroSAU]{САУ}, \textit{k= 1,2,...} - номер канала управления, обеспечивающего качество изображения, М – показатель колебательности, $\varDelta \varphi$, $\varDelta L$ –запасы устойчивости по фазе и по амплитуде, полученные из логарифмической частотной характеристики  разомкнутой системы \cite[]{Bessekerski20}.

Согласно предлагаемой методике алгоритм разработки СВ, \hyperref[acroSAU]{САУ} и исследования их динамики представлен в виде 4-х основных последовательных интерактивных замкнутых процедур с использованием 27-ми блоков разработки и исследования (рисунок~\ref{fig:tikz_example}):

\begin{enumerate}
	\item После выполнения последовательных процедур (верификации, разработки расчетной и математической моделей, декомпозиции) с применением компьютерных технологий (Solid Works, MathCAD, MATLAB): (рисунок~\ref{fig:tikz_example}: 1,2,...7) проводится интерактивный синтез алгоритмов управления изолированных каналов \hyperref[acroSAU]{САУ} частотным методом \cite[]{Bessekerski20} и конструктивных параметров СВ (рисунок~\ref{fig:tikz_example}: 8-9-21-8) до выполнения критериев качества (глава \ref{ch:ch4}). На основании полученной информации проводится разработка и изготовление масштабного динамического макета \hyperref[acroEOS]{ОЭП} (ОУ, приводов, встроенных датчиков с обеспечением адекватности их динамическим характеристикам) с применением 3Д - принтера (рисунок~\ref{fig:tikz_example}: 10). Далее проводится исследование динамики пространственного движения макета \hyperref[acroSAU]{САУ} и СВ с использованием синтезированных алгоритмов управления (рисунок~\ref{fig:tikz_example}: 11-12). Если требования ТЗ (рисунок~\ref{fig:tikz_example}: 2) не выполняются, то проводятся последовательно циклы итерационных исследований динамики \hyperref[acroSAU]{САУ} и СВ : (рисунок~\ref{fig:tikz_example}: 12-20-10,...12), (рисунок~\ref{fig:tikz_example}: 12-20-21-8,...12), (рисунок~\ref{fig:tikz_example}: 12-20-23-3,...12) до обеспечения критериев качества (глава \ref{ch:ch4}), заключающихся в оптимальном выборе параметров регулятора и ОУ.
	\item Далее переходим к исследованию пространственной модели с применением MATLAB: 
	(рисунок~\ref{fig:tikz_example}: 13 – 14 - 15). Если критерии (глава \ref{ch:ch4}) не выполняются, то переходим на 2-е круги последовательных итераций: (рисунок~\ref{fig:tikz_example}: 15-24-21-8,...15), (рисунок~\ref{fig:tikz_example}: 15-24-23-3,...15), где доопределяем число степеней свободы математической модели и её параметры, параметры \hyperref[acroSAU]{САУ} и СВ путем последовательных предыдущих итераций исследований до выполнения критериев (глава \ref{ch:ch4}). C учетом полученной информации доопределяем: необходимые конструктивные доработки  ОУ, \hyperref[acroSAU]{САУ}, СВ ; приемлемые варианты построения СВ и \hyperref[acroSAU]{САУ}, а также в случае необходимости уточняем критерии (глава \ref{ch:ch4}) и задачи, которые могут решать СВ и \hyperref[acroSAU]{САУ}, а также ограничения и нелинейности в контурах управления. По результатам исследования делается заключение о необходимости изготовления опытного образца \hyperref[acroSAU]{САУ} и СВ или проведения дальнейших исследований.
	
	\item Затем переходим к испытаниям опытного образца \hyperref[acroSAU]{САУ} и СВ на стендах в соответствии с методиками испытаний и требований ТЗ: (рисунок~\ref{fig:tikz_example}: 16-17). Если критерии (глава \ref{ch:ch4}) не выполняются, то переходим  на 
	3-и круги последовательных итерационных процедур: (рисунок~\ref{fig:tikz_example}: 17-25-21-8,...17), (рисунок~\ref{fig:tikz_example}: 17-25-26-1,...17), которые включает в себя две предыдущих. Результаты исследований фиксируют в протоколе испытаний и делают заключение о необходимости доработок СВ и \hyperref[acroSAU]{САУ} или допуске их к испытаниям на борту. 
	
	\item В заключении переходим к испытаниям \hyperref[acroSAU]{САУ} и СВ на борту в соответствии с методиками натурных испытаний и требований ТЗ: (рисунок~\ref{fig:tikz_example}: 18-19). Если критерии (глава \ref{ch:ch4}) не выполняются, то переходим  на 
	4-и круги последовательных итерационных процедур: (рисунок~\ref{fig:tikz_example}: 19-27-23-3,...19), (рисунок~\ref{fig:tikz_example}: 19-27-26-1,...19), которые включают в себя (при необходимости) три предыдущие. Результаты испытаний и требования к техническим характеристикам СВ и \hyperref[acroSAU]{САУ} фиксируем в протоколе испытаний и делаем заключение о необходимости доработок СВ и \hyperref[acroSAU]{САУ} или допуске их к дальнейшему производству.
	Приведенная методика была апробирована при разработке ряда \hyperref[acroSAU]{САУ} \hyperref[acroEOS]{ОЭП} \cite[]{Belyakov}, \cite[]{Karpov}, \cite[]{Baloev16}, \cite[]{Karpov17}, \cite[]{Gerasin19}, \cite[]{Molin21}. Каждому из блоков на рисунке ниже (рисунок~\ref{fig:tikz_example}) присущи своя специфика и его математическое или логическое описание и предполагается соответствующая методика его разработки. Некоторые из них приводятся ниже.

\end{enumerate}

\section{Оценка допуска на точность стабилизации изображения} \cite[]{Belyakov}, \cite[]{Sokolski22}, \cite[]{Molin21} \label{sec:ch2/sec2} 



Для изучения процесса формирования изображения с учетом множества факторов, влияющих на формирование, преобразование и передачу качества изображения, рассмотрим функциональную схему одного из вариантов комплексированного \hyperref[acroEOS]{ОЭП} (рисунок~\ref{fig:oep_sch}), представляющего собой совокупность \hyperref[acroTVS]{тепловизионной системы (ТВС)}, \hyperref[acroTS]{телевизионной системы (ТС)}, \hyperref[acroFS]{фотографической системы (ФС)}, наблюдательных приборов в видимой и ближней инфракрасной (ИК) областях. В результате действия внешних возмущений $F(P,g,T,t)$, возмущений, идущих от носителя, и наличия управления \hyperref[acroEOS]{ОЭП} в пространстве остается не компенсированный сдвиг изображения (динамическая погрешность).


\begin{figure}[ht]
	{\centering
		\ifdefmacro{\tikzsetnextfilename}{\tikzsetnextfilename{tikz_example_compiled}}{}% присваиваемое предкомпилированному pdf имя файла
		% !TEX encoding = UTF-8 Unicode
% Úτƒ-8 encoded
% http://www.linux.org.ru/forum/general/10357036
\tikzset{
    line/.style={draw, -latex'},
    every join/.style={line},
    u/.style={anchor=south},
    r/.style={anchor=west},
    fxd/.style={text width = 6em},
    it/.style={font={\small\itshape}},
    bf/.style={font={\small\bfseries}}
}
\tikzstyle{base} =
    [
        draw,
        on chain,
        on grid,
        align=center,
        minimum height=4ex,
        minimum width = 10ex,
        node distance = 6mm and 60mm,
        text badly centered
    ]
\tikzstyle{coord} =
    [
        coordinate,
        on chain,
        on grid
    ]
\tikzstyle{cloud} =
    [
        base,
        ellipse,
        fill = red!5,
        node distance = 3cm,
        minimum height = 2em
    ]
\tikzstyle{decision} =
    [
        base,
        diamond,
        aspect=2,
        fill = green!10,
        node distance = 2cm,
        inner sep = 0pt
    ]
\tikzstyle{block} =
    [
        rectangle,
        base,
        fill = blue!3,
        rounded corners,
        minimum height = 2em
    ]
\tikzstyle{print_block} =
    [
        base,
        tape,
        tape bend top=none,
        fill = yellow!10
    ]
\tikzstyle{io} =
    [
        base,
        trapezium,
        trapezium left angle = 70,
        trapezium right angle = 110,
        fill = blue!5
    ]
\makeatletter
\pgfkeys{/pgf/.cd,
    subrtshape w/.initial=2mm,
    cycleshape w/.initial=2mm
}
\pgfdeclareshape{subrtshape}{
    \inheritsavedanchors[from=rectangle]
    \inheritanchorborder[from=rectangle]
    \inheritanchor[from=rectangle]{north}
    \inheritanchor[from=rectangle]{center}
    \inheritanchor[from=rectangle]{west}
    \inheritanchor[from=rectangle]{east}
    \inheritanchor[from=rectangle]{mid}
    \inheritanchor[from=rectangle]{base}
    \inheritanchor[from=rectangle]{south}
    \backgroundpath{
        \southwest \pgf@xa=\pgf@x \pgf@ya=\pgf@y
        \northeast \pgf@xb=\pgf@x \pgf@yb=\pgf@y
        \pgfmathsetlength\pgfutil@tempdima{\pgfkeysvalueof{/pgf/subrtshape w}}
        \def\ppd@offset{\pgfpoint{\pgfutil@tempdima}{0ex}}
        \def\ppd@offsetm{\pgfpoint{-\pgfutil@tempdima}{0ex}}
        \pgfpathmoveto{\pgfqpoint{\pgf@xa}{\pgf@ya}}
        \pgfpathlineto{\pgfqpoint{\pgf@xb}{\pgf@ya}}
        \pgfpathlineto{\pgfqpoint{\pgf@xb}{\pgf@yb}}
        \pgfpathlineto{\pgfqpoint{\pgf@xa}{\pgf@yb}}
        \pgfpathclose
        \pgfpathmoveto{\pgfpointadd{\pgfpoint{\pgf@xa}{\pgf@yb}}{\ppd@offsetm}}
        \pgfpathlineto{\pgfpointadd{\pgfpoint{\pgf@xa}{\pgf@ya}}{\ppd@offsetm}}
        \pgfpathlineto{\pgfpointadd{\pgfpoint{\pgf@xb}{\pgf@ya}}{\ppd@offset}}
        \pgfpathlineto{\pgfpointadd{\pgfpoint{\pgf@xb}{\pgf@yb}}{\ppd@offset}}
        \pgfpathclose
    }
}
\pgfdeclareshape{cyclebegshape}{
    \inheritsavedanchors[from=rectangle]
    \inheritanchorborder[from=rectangle]
    \inheritanchor[from=rectangle]{north}
    \inheritanchor[from=rectangle]{center}
    \inheritanchor[from=rectangle]{west}
    \inheritanchor[from=rectangle]{east}
    \inheritanchor[from=rectangle]{mid}
    \inheritanchor[from=rectangle]{base}
    \inheritanchor[from=rectangle]{south}
    \backgroundpath{
        \southwest \pgf@xa=\pgf@x \pgf@ya=\pgf@y
        \northeast \pgf@xb=\pgf@x \pgf@yb=\pgf@y
        \pgfmathsetlength\pgfutil@tempdima{\pgfkeysvalueof{/pgf/cycleshape w}}
        \pgfpathmoveto{\pgfqpoint{\pgf@xa}{\pgf@ya}}
\pgfpathlineto{\pgfpointadd{\pgfpoint{\pgf@xa}{\pgf@yb}}{\pgfpoint{0ex}{-\pgfutil@tempdima}}}
\pgfpathlineto{\pgfpointadd{\pgfpoint{\pgf@xa}{\pgf@yb}}{\pgfpoint{\pgfutil@tempdima}{0ex}}}
\pgfpathlineto{\pgfpointadd{\pgfpoint{\pgf@xb}{\pgf@yb}}{\pgfpoint{-\pgfutil@tempdima}{0ex}}}
\pgfpathlineto{\pgfpointadd{\pgfpoint{\pgf@xb}{\pgf@yb}}{\pgfpoint{0ex}{-\pgfutil@tempdima}}}
\pgfpathlineto{\pgfqpoint{\pgf@xb}{\pgf@ya}}
        \pgfpathclose
    }
}
\pgfdeclareshape{cycleendshape}{
    \inheritsavedanchors[from=rectangle]
    \inheritanchorborder[from=rectangle]
    \inheritanchor[from=rectangle]{north}
    \inheritanchor[from=rectangle]{center}
    \inheritanchor[from=rectangle]{west}
    \inheritanchor[from=rectangle]{east}
    \inheritanchor[from=rectangle]{mid}
    \inheritanchor[from=rectangle]{base}
    \inheritanchor[from=rectangle]{south}
    \backgroundpath{
        \southwest \pgf@xa=\pgf@x \pgf@ya=\pgf@y
        \northeast \pgf@xb=\pgf@x \pgf@yb=\pgf@y
        \pgfmathsetlength\pgfutil@tempdima{\pgfkeysvalueof{/pgf/cycleshape w}}
        \pgfpathmoveto{\pgfqpoint{\pgf@xb}{\pgf@yb}}
\pgfpathlineto{\pgfpointadd{\pgfpoint{\pgf@xb}{\pgf@ya}}{\pgfpoint{0ex}{\pgfutil@tempdima}}}
\pgfpathlineto{\pgfpointadd{\pgfpoint{\pgf@xb}{\pgf@ya}}{\pgfpoint{-\pgfutil@tempdima}{0ex}}}
\pgfpathlineto{\pgfpointadd{\pgfpoint{\pgf@xa}{\pgf@ya}}{\pgfpoint{\pgfutil@tempdima}{0ex}}}
\pgfpathlineto{\pgfpointadd{\pgfpoint{\pgf@xa}{\pgf@ya}}{\pgfpoint{0ex}{\pgfutil@tempdima}}}
\pgfpathlineto{\pgfqpoint{\pgf@xa}{\pgf@yb}}
        \pgfpathclose
    }
}
\makeatother
\tikzstyle{subroutine} =
    [
        base,
        subrtshape,
        fill = green!25
    ]
\tikzstyle{cyclebegin} =
    [
        base,
        cyclebegshape,
        fill = blue!25
    ]
\tikzstyle{cycleend} =
    [
        base,
        cycleendshape,
        fill = blue!25
    ]
\tikzstyle{connector} =
    [
        base,
        circle,
        fill = red!25
    ]

\begin{tikzpicture}[%
    start chain=going below,    % General flow is top-to-bottom
    node distance=6mm and 150mm, % Global setup of box spacing
        ]
        \node [cloud] (start) {Начало};
        \node [block, join] (phase1) {1,2};
        \node [block, join] (phase3) {3};
        \node [block, join] (phase4) {4567};
        \node [block, join] (phase8) {8};
        \node [block, join] (phase9) {9};
        \node [block, join] (phase10) {10};
        \node [block, join] (phase11) {11};
        \node [decision, join, right of = phase11, node distance = 4cm] (condition12) {12};
        \node [block, join, left of = condition12, node distance = 4cm] (phase13) {13,14};
        \node [decision, join, right of = phase13, node distance = 4cm] (condition15) {15};
        \node [block, join, left of = condition15, node distance = 4cm] (phase16) {16};
        \node [decision, join, right of = phase16, node distance = 4cm] (condition17) {17};
        \node [block, join, left of = condition17, node distance = 4cm] (phase18) {18};
        \node [decision, right of = phase18, node distance = 4cm] (condition19) {19};
        \node [cloud] (fin) {Конец};
        
        \node [block, right of = phase1, node distance = 4cm] (print26) {26};
        \node [block, right of = phase3, node distance = 4cm] (print23) {23};
        \node [block, right of = phase8, node distance = 4cm] (print21) {21};
        \node [decision, right of = condition12, node distance = 4cm] (print20) {20};
        \node [decision, right of = condition15, node distance = 4cm] (print24) {24};
        \node [decision, right of = condition17, node distance = 4cm] (print25) {25};
        \node [decision, right of = condition19, node distance = 4cm] (print27) {27};
        
        \path [line, red] (condition12) -| node [u,near start] {Нет} (print20);
        \path [line, red] (condition15) -| node [u,near start] {Нет} (print24);
        \path [line, red] (condition17) -| node [u,near start] {Нет} (print25);
        \path [line, red] (condition19) -| node [u,near start] {Нет} (print27);
        
        \path [line, green] (condition19) to node [r] {Да}(fin);

\end{tikzpicture}

		
	}
	\legend{}
	\caption[Пример \texttt{tikz} схемы]{Пример рисунка, рассчитываемого
		\texttt{tikz}, который может быть предкомпилирован}
	\label{fig:tikz_example}
\end{figure}


Это приводит к изменению структуры изображения аналогично влиянию аберраций, т.е. к ухудшению качества изображения, зависящего от величины и характера изменения динамических погрешностей управляющих систем. Для оценки допуска на точность стабилизации изображения используют частотный критерий качества изображения \hyperref[acroEOS]{ОЭП} – функцию передачи модуляции (\hyperref[acroFPM]{ФПМ}) \cite[]{Tarasov}.

\begin{figure}[ht]
	\centering
	\includegraphics[width=0.8\linewidth]{oep_sch} 
	\caption{Функциональная схема обобщённой модели \hyperref[acroEOS]{ОЭП}}
	\label{fig:oep_sch}
\end{figure}

Обозначения: 
\begin{itemize}
	\item $F(P,g,T,t)$ – функция возмущений, где
	\begin{itemize}
		\item $P$ – атмосферное давление,
		\item $g$ – земное ускорение,
		\item $T$ - температура, 
		\item $t$ – время;
	\end{itemize}
	\item ФП – фотоприемник, 
	\item ЭОП – электронно-оптический преобразователь, 
	\item ССк – система сканирования, 
	\item У-П – усилитель-преобразователь, 
	\item СКИ – система коррекции изображения.
\end{itemize}

Оценка качества изображения управляемых \hyperref[acroEOS]{ОЭП} определяется на основе анализа произведения \hyperref[acroFPM]{ФПМ} элементов оптико-электронного тракта, формирующих изображение и влияющих на качество изображения. В рамках принятой структуре \hyperref[acroEOS]{ОЭП} его \hyperref[acroFPM]{ФПМ} должна удовлетворять по каждому каналу визуализации условию \cite[]{Ivanov18}, \cite[]{Molin21}.

\begin{equation}
\label{eq:p2:2.5}
\begin{alignedat}{2}
T_{\textit{ОЭП}}(N)=
T_{\textit{ат}}(N)T_{\textit{об}}(N)T_{\textit{фп}}(N)T_{\textit{пои}}(N)T_{\textit{САУ}}(N)>
T_{\textit{ОЭC}}^{\textit{доп}}(N)
\end{alignedat}
\end{equation}

\begin{equation}
\label{eq:p2:3}
\begin{alignedat}{2}
T_{\textit{ОЭC}}^{\textit{доп}}\left(\nu{}\right)=
\frac{N{\gamma{}}_\textit{и}}{2\sin{\left(N\frac{{\gamma{}}_\textit{и}}{2}\right)}}
\exp{\left[-\frac{{\left(N{\gamma{}}_p\right)}^2}{16\ln{\left(2m\right)}}\right]}
\end{alignedat}
\end{equation}

\begin{equation}
\label{eq:p2:4}
\begin{alignedat}{2}
T_{\Sigma{}}^{\textit{доп}}\left(N\right)=\exp{\left(-2{\pi{}}^2{\sigma{}}^2N^2\right)}
\end{alignedat}
\end{equation}

где 
$Т_{\textit{оэп}}(N)$ – \hyperref[acroFPM]{ФПМ}  \hyperref[acroEOS]{ОЭП}, 
$N$ – пространственная частота, 
$m$ - отношение сигнал/шум, 
$\gamma_\textit{и}$ - угловой размер источника излучения, 
$\gamma_\textit{p}$ - угловое разрешение  \hyperref[acroEOS]{ОЭП}, 
$T_{\textit{ОЭП}}^{\textit{доп}}(N)$ – допустимая \hyperref[acroFPM]{ФПМ}  \hyperref[acroEOS]{ОЭП}, 
$Т_\textit{ат}(N)$ – \hyperref[acroFPM]{ФПМ} атмосферы, 
$T_\textit{oб}(N)$ – \hyperref[acroFPM]{ФПМ} объектива, 
$Т_\textit{фп}(N)$ – \hyperref[acroFPM]{ФПМ} фотоприемника, 
$Т_\textit{пои}(N)$ – \hyperref[acroFPM]{ФПМ} преобразования оптической информации, вид которых можно найти  в \cite[]{Tarasov}, 
$T_{\textit{САУ}}(N)$ – \hyperref[acroFPM]{ФПМ} сдвига изображения (динамической погрешности \hyperref[acroSAU]{САУ}), зависящая от вида динамического смещения изображения: линейного (Л) - $\dot{x}(t) = Vt$, 
гармонического (Г) - $x(t)=a_{0}\sin{\omega t}$ и 
случайного (СЛ) \cite[]{Tarasov}, \cite[]{Sokolski22}.

Тогда в частном случае допустимая \hyperref[acroFPM]{ФПМ} \hyperref[acroSAU]{САУ} (системы слежения (Л, Г), системы виброзащиты (СВ), автоматической фокусировки (САФ)) определится соответственно: 

\begin{equation}
\label{eq:p2:6}
\begin{alignedat}{2}
T_{\textit{САУ}}\left(N\right)=
T_\textit{Л}\left(N\right)T_\textit{Г}\left(N\right)T_{\textit{СВ}}\left(N\right)T_{\textit{САФ}}\left(N\right)\geq{}T_{\textit{САУ}}^{\textit{доп}}(N)= \\
\dfrac{ T_{\textit{оэп}}^{\textit{доп}}(N) }{ T_{\textit{ат}}(N)T_{\textit{об}}(N)T_{\textit{фп}}(N)T_{\textit{пои}}(N) }
\end{alignedat}
\end{equation}

\begin{equation}
\label{eq:p2:7}
\begin{alignedat}{3}
T_\textit{Л}\left(N\right)=Sinc(\pi{}VtN) , \\
T_{\textit{Г}}(N)=J_0\left(2\pi{}a_0N\right) , \\
T_{\textit{СВ}}\left(N\right) = exp[-2{(\pi{}a_{\textit{ср}}N)}^2]
\end{alignedat}
\end{equation}

где $T_{\textit{САФ}}(N)$ – ФПМ САФ; 
$V$ – допустимая скорость движения изображения; 
$J0()$ - функция Бесселя нулевого порядка, 
$a_0$ –допустимая амплитуда гармонических колебаний,   
$a_{\textit{ср}}$ – среднее значение допустимой амплитуды случайного сдвига изображения. 
Выражения для оценки допустимых динамических погрешностей для ССАУ и СВ можно найти в \cite[]{Karpov}, \cite[]{Karpov23}. Допуск на точность САФ для фотообъективов в видимой области спектра, инфракрасных систем можно найти в \cite[]{Tarasov}, \cite[]{Belyakov}.

Пространственная расчетная частота Np должна соответствовать средним пространственным частотам (частоте Найквиста ($N_{\textit{н}} (\textit{мм}^{-1})$ либо $f_H (\textit{рад}^{-1})$), которая выбирается c учетом условий \cite[]{Tarasov}: 

\begin{equation}
\label{eq:p2:8}
\begin{alignedat}{2}
N_H=0.5 N_{\textit{выб}}\leq{}N_{\textit{гр}},
N_{\textit{гр}}=\frac{D}{1.22\lambda{}f^{'}},\\
f_H=\frac{X_\textit{э}}{2f^{'}},
f_{\textit{гр}}=\frac{D}{\lambda{}}(\textit{рад}^{-1}),
f_{\textit{выб}}=\frac{n_{\textit{э}}}{2\omega{}},\\
D\geq{}\frac{1.22 k_{\textit{аб}}\lambda{}}{h_{\textit{кр}}},
N_x=\frac{N_{\textit{д}}l}{h_{\textit{кр}}f^{'}},\\
f_x=\frac{N_{\textit{д}}l}{h_{\textit{кр}}},
f_{xy}=\frac{N_{\textit{д}}l}{\sqrt{h_{\textit{крx}}h_{\textit{крy}}}}=\sqrt{f_x^2+f_y^2},\\
\Delta{}{\omega{}}_p=\frac{2h_{\textit{кр}}}{N_{\textit{д}}l},
\Delta{}{\omega{}}_{\textit{аб}} =\frac{2.44 k_{\textit{аб}}\lambda{}}{D},
\Delta{}{\omega{}}_{\textit{э}} =\frac{d_{\textit{э}}}{f^{'}},\\
\frac{D}{f^{'}}=\frac{2.44 k_{\textit{аб}}\lambda{}}{d_{\textit{э}}},\\
\end{alignedat}
\end{equation}

где 
$N_{\textit{гр}}$ – граничная пространственная частота, 
$f_{\textit{выб}}$ – частота выборки, 
$X_{\textit{э}} ,n_{\textit{э}}$ –период расположения и число элементов фотоприемника, 
$2\omega$ - поле зрения ОЭП, 
$k_{\textit{аб}}$ – коэффициент, учитывающий аберрации объектива, 
$h_{\textit{кр}}$ – критический размер объекта наблюдения (ОН), 
$l$ – расстояние до ОН, 
$N_{x} , f_{x}$ – пространственные частоты, соответствующие критериям Джонсона, 
$N_{\textit{д}}$ – числа элементов разрешения (критерии Джонсона – обнаружения, классификации, распознавания и идентификации), 
$f_{xy}$ - пространственная частота Джонсона в двух ортогональных направлениях \textit{x} и \textit{y}, 
$\Delta{}{\omega{}}_p$ – требуемое геометрическое угловое разрешение, необходимое для наблюдения, 
$\Delta{}{\omega{}}_{\textit{аб}}$ – минимальное угловое значение кружка рассеяния объектива с учетом аберраций, 
$\Delta{}{\omega{}}_{\textit{э}} , d_{\textit{э}}$ – угловой и линейный размеры элемента чувствительного слоя приемника излучения. 

Если при $N \ge N_p$ качество изображения ОЭС будет удовлетворять условию $Т(N) \ge 0.8$, то оно считается хорошим. Таким образом, используя  (\labelcref{eq:p2:6,eq:p2:7,eq:p2:8}), можно определить допустимые динамические погрешности, обеспечивающие выполнение условия (\ref{eq:p2:6}).

\section{Разработка математической модели} \label{sec:ch2/sec3}

При разработке ОЭП одной из важнейших задач является управление направлением линии визирования в пространстве. Оно осуществляется, как правило, двумя способами: путем управления всем устройством информационных каналов (ОЭП 1-го типа) и управлением положением отдельных оптических элементов (зеркал, призм) (ОЭП 2-го типа) \cite[]{Karpov23}. Оба способа построения управляющих ОЭП имеют свои особенности и широко применяются.

Для построения математических моделей относительного движения ОЭП (движения относительно корпуса летательного аппарата (ЛА), на котором они устанавливаются) используются уравнения Лагранжа II-го рода. За инерциальную систему отсчета принимается система координат, связанная с поверхностью Земли. Для записи уравнений Лагранжа II-го рода используется смешанный метод Жильбера \cite[]{Belyakov}, \cite[]{Baloev16}. Для изучения динамики ОЭП 1-го типа моделируется двумя абсолютно твердыми телами: «вилка» - азимутальный блок, в которой установлено второе тело - оптико-электронный блок (ОЭБ). Оси вращения «вилки» и ОЭБ взаимно перпендикулярны и пересекаются. Угол поворота «вилки» по азимуту – $\alpha$, угол поворота ОЭБ по углу места –$\beta$. Математическая модель относительного движения ОЭП 1-го типа определяется следующим матричным уравнением \cite[]{Karpov23}:

\begin{equation}
\label{eq:p2:6-}
\begin{alignedat}{2}
A_c\left(q_c\right){\ddot{q}}_c+B_c\left(t,q_c\right){\dot{q}}_c+Q_c\left(t,q_c\right)+\\
F_c\left(t,q_c\right)+F1_c\left(q_c,{\dot{q}}_c\right)=\frac{c_M}{r}u-M_{\textit{тр}}
\end{alignedat}
\end{equation}

где
$q_c=\left(\begin{array}{cc}
\alpha{} \\
\beta{}
\end{array}\right)$,      
$u=\left(\begin{array}{
	cc}
u_1 \\
u_2
\end{array}\right)$

$M_{\textit{тр}}=\left(\begin{array}{
		cc}
	M_{\textit{тр.1}}sign\left(\dot{\alpha{}}\right) \\
	M_{\textit{тр.2}}sign\left(\dot{\beta{}}\right)
\end{array}\right)$, 
$A_c\left(q_c\right),B_c\left(t,q_c\right)$ – матрицы-функции  размерности $2 \times 2$,
$Q_c\left(t,q_c\right),F_c\left(t,q_c\right),F1_c\left(q_c,{\dot{q}}_c\right)$ – столбцы-функции размерности $2 \times 1$,
$u_1$, $u_2$ - управляющее сигналы на моментных двигателях по азимуту и углу места. 

Азимутальный блок и угломестный блок ОЭП 2-го типа совместно с соосными приводами моделируется двумя подвижными твердыми телами, установленными в кардановом подвесе прибора, который закреплен на ЛА. Положение этих тел относительно корпуса однозначно определяется углами поворотов азимутального ($\phi_1$) и угломестного ($\phi_2$) блоков. Математическая модель относительного движения ОЭП 2-го типа определяется следующим матричным уравнением \cite[]{Baloev16}:

\begin{equation}
\label{eq:p2:7-}
\begin{alignedat}{2}
A\left(\phi{}\right)\ddot{\phi{}}+N\left(\phi{},\dot{\phi{}}\right)+H\left(\phi{},{\omega{}}_1\right)\dot{\phi{}}+\\
L\left(\phi{}\right){\epsilon{}}_1(t)-\Omega{}\left(\phi{},{\omega{}}_1\right)=\frac{c_M}{r}u-M_{mp}-P\left(\phi{},a\right)
\end{alignedat}
\end{equation}

$\phi{}=\left(\begin{array}{
	cc}
{\phi{}}_1 \\
{\phi{}}_2
\end{array}\right)$,     $u=\left(\begin{array}{
	cc}
u_1 \\
u_2
\end{array}\right),$   $M_{mp}=\left(\begin{array}{
	cc}
M_{\textit{mp.1}} \dot{\phi_1}   \\
M_{\textit{mp.2}} \dot{\phi_2}
\end{array}\right)$,

$A\left(\phi{}\right),H\left(\phi{},{\omega{}}_1\right)$ – матрицы-функции  размерности $2 \times 2$, $L(\phi)$– матрица-функция размерности $2 \times 3$,
$P\left(\phi{},a\right),N\left(\phi{},\dot{\phi{}}\right),\Omega{}\left(\phi{},{\omega{}}_1\right)$ –столбцы-функции размерности $2 \times 1$,
${\omega{}}_1(t)=\left(\begin{array}{
	ccc}
{\omega{}}_{X1}(t) \\
{\omega{}}_{Y1}(t) \\
{\omega{}}_{Z1}(t)
\end{array}\right)$
,    ${\epsilon{}}_1(t)=\left(\begin{array}{
	ccc}
{\dot{\omega{}}}_{X1}(t) \\
{\dot{\omega{}}}_{Y1}(t) \\
{\dot{\omega{}}}_{Z1}(t)
\end{array}\right)=\left(\begin{array}{
	ccc}
{\epsilon{}}_{X1}(t) \\
{\epsilon{}}_{Y1}(t) \\
{\epsilon{}}_{Z1}(t)
\end{array}\right)$
– векторы, составленные из проекций векторов угловой скорости и углового ускорения ЛА  на оси, связанные с корпусом ОЭП. Значения элементов матриц-функций определяется величинами осевых и центробежных моментов инерции, положением центров масс и массами тел прибора. В уравнениях (\labelcref{eq:p2:6-,eq:p2:7-,}) зависимость от времени определяется уравнениями движения ЛА ($\omega_1(t)$).

\section{Верификация параметров} \label{sec:ch2/sec4}

Одним из наиболее эффективных методов построения математических моделей является экспериментальный метод – верификация (идентификация). Суть идентификации состоит в том, что по реакции $x(t)$ исследуемой САУ (или её части) на известные виды воздействий $у(t)$ определяют динамические характеристики САУ (или ее части), а по ним и ее математическую модель. Процедура идентификации по переходным и частотным характеристикам САУ с заданным критерием адекватности реальной системе включает в себя интерактивный процесс построения расчетных и математических моделей САУ \cite[]{Karpov}. Пусть:

\begin{enumerate}
	\item Модель САУ управляема и наблюдаема \cite[]{Bessekerski20} и описывается системой 
	\begin{equation}
	\label{eq:p2:8-}
	\begin{alignedat}{2}
	\alpha{}={\left(E+W(P)\right)}^{-1}W(p)y
	\end{alignedat}
	\end{equation}
	где $W_{ij}(р)={\left\Vert{}W_{ij}(р)\right\Vert{}}_{rxr}$	
	передаточная матрица разомкнутой системы, 
	$E$ – единичная матрица,
	$\alpha$ - вектор регулируемых координат, 
	$у$ - вектор управления, 
	$r$ – число регулируемых координат. 
	Обозначим переходные процессы исходной модели ($М_I$) и последующих моделей ($M_k$, k=2,3...), 
	определяемых далее в процессе идентификации, при известных $\alpha_k (t_0 ), у_k (t)$ через 
	$\alpha_{i} (t)={\left\Vert{} \alpha_{i}(v^k,t) \right\Vert{}}_{rx1}$, 
	а частотные характеристики (ЧХ) разомкнутых САУ через 
	$W_{ij}(j\omega) = {\left\Vert{} W^k_{ij}(v^k, j\omega) \right\Vert{}}_{rxr}$.  
	Здесь и далее: $k$ – номер  модели, $v^k$ – вектор варьируемых параметров k–ой  модели, $D_{v^k}$ - область существования параметров
	\begin{equation}
	\label{eq:p2:9-}
	\begin{alignedat}{2}
	v^k \in D_{v^k} (k= 1,2,...)
	\end{alignedat}
	\end{equation}	
	\item Получены ЧХ элементов матрицы 
	$\tilde{W}_{ij}(j\omega) = {\left\Vert{} \tilde{W}_{ij}( j\omega) \right\Vert{}}_{rxr}$
	реальной САУ в рабочем диапазоне частот 
	$\Omega{}=\left[{\omega{}}_1,{\omega{}}_2\right]$ 
	разомкнутых каналов управления и переходные процессы 
	$\tilde{\alpha}_{i}(t) = {\left\Vert{} \tilde{\alpha}_{i}(i) \right\Vert{}}_{rx1}$
	в процессе нормальной эксплуатации при заданных 
	$\alpha(t_0)$, $y(t)$ за время $T=[t_0,t_1]$.
	
	Оценка степени адекватности модели и идентичности каждой полученной модели реальной системе, а также уточнение моделей может оцениваться следующими критериями адекватности: 
	
	\begin{equation}
	\label{eq:p2:10-}
	\left. %ВАЖНО: точка после слова left делает скобку неотображаемой
	\begin{aligned}
	\min_v \max_\omega \left| L_m\left| \tilde{W}_{ij}(j\omega)  \right| - L_m\left| {W}^k_{ij}(v^k, j\omega)  \right| \right| &\leq \varDelta L(\omega) \\
	\min_v \max_\omega \left| arg \tilde{W}_{ij}(j\omega) - arg {W}^k_{ij}(v^k, j\omega) \right| &\leq \varDelta \phi(\omega) \\
	\dfrac{1}{T}\int^T_0\left| \alpha_i(t,v^k)- \tilde{\alpha}_i(t) \right| dt &\leq \epsilon_i
	\end{aligned}\right\}
	\end{equation}
	где $Lm(x)=20lg(x)$, $\varDelta L(\omega)$, $\varDelta \phi(\omega)$, $\epsilon_i$ – невязки, характеризующие погрешность идентификации. Невязки $\varDelta L(\omega)$, $\varDelta \phi(\omega)$, $\epsilon_i$ выбираются, исходя из допустимых запасов устойчивости и требуемого качества переходного процесса, определяемых в соответствии с ТЗ на САУ.
	
\end{enumerate}

В основе построения динамических моделей и математического описания их поведения лежит интерактивный подход, базирующийся на сочетании использования физических законов с экспериментом. На каждом этапе исследования выбирается соответствующий уровень идеализации модели. Причем модель и ее математическое описание может на основе дополнительно получаемой информации уточняться и изменяться. При оценке критериев адекватности необходимо решать задачу оптимизации параметров в ограниченной области (\labelcref{eq:p2:9-}) по критериям (\labelcref{eq:p2:10-}). 
Исходя из требований, предъявляемых к САУ, и априорной информации, выбирается динамическая модель первого приближения ($МI$), где допускается $m_1 < n$, $m_1$ - число обобщенных координат $М1$, $n$ – число обобщенных координат полной модели. Для $М1$ составляются дифференциальные уравнения движения. 
Анализируются плоские задачи путем детального изучения динамики сепаратных каналов по отдельным ступеням: исследуются частотный спектр и чувствительность собственных частот к изменению параметров, выясняется влияние возмущений, строятся ЧХ и переходные процессы, выявляются существенные нелинейности и т.д. 
Часто по требованиям к динамической системе строится желаемая математическая модель, затем путем сравнения этой модели с исследуемой подыскиваются физически реализуемые элементы модели. В случае, если исследуемая САУ не разделяется на плоские задачи, то по результатам детального исследования плоских задач составляется пространственная динамическая модель системы. Выводятся уравнения движения в пространстве (в режиме переориентации – нелинейные, в режиме стабилизации – линейные) с учетом существенных нелинейностей, Коэффициенты уравнений выражаются через механические и электрические параметры системы. 
Анализируя модель МI, находим ее  
ЧХ - $W_{ij} (v^1,j\omega),\alpha_i (v^1,t) (i,j=1,...,r)$ для исходных значений 
$v^1 \in \tilde{D}_{v^1}$. Варьируя параметрами МI известными частотными методами \cite[]{Bessekerski20}, отыскиваются параметры модели из области (\labelcref{eq:p2:8-}) такими, чтобы выполнялись частотные условия  - 1,2 критерия (\labelcref{eq:p2:10-}). Если же за счет выбора параметров условия (\labelcref{eq:p2:10-}) выполнить не удается, то путем дополнительных исследований анализируется имеющаяся и новая информация $ W(j\omega,\Omega{}) $  о поведении ОУ и на ее основе строится модель второго приближения (М2) (меняется структура ОУ), где число обобщенных координат $m_2>m_1$, $m_2<n$. И далее проводится ее исследование. 
В случае же недостаточности информации для построения М2 планируем эксперимент для выявления более "тонких" динамических свойств реальной системы. Далее, аналогично предыдущему с учетом ограничений на параметры (\labelcref{eq:p2:9-}) доопределяем модель М2 согласно (\labelcref{eq:p2:9-}). Если же опять путем выбора параметров условие (\labelcref{eq:p2:10-}) не удается выполнить, то процесс построения модели продолжается далее (М3, М4 и т.д.) до выполнения частотных критериев (\labelcref{eq:p2:10-}). После этого проверяем 3-е условие (\labelcref{eq:p2:10-}). Если идентифицированная по ЧХ модель САУ не удовлетворяет условиям (\labelcref{eq:p2:10-}), то при построении следующих моделей аналогично предыдущему отыскиваем такие структуры и параметры моделей, которые удовлетворяют всем трем условиям (\labelcref{eq:p2:10-}). 
Таким образом, интерактивный процесс построения динамической модели и математического описания ее поведения включает выбор динамической модели, составление уравнений движения, формулировку критериев оценки степени адекватности ее реальной системе и отыскание адекватной модели согласно этим критериям.


\section{Оценки декомпозируемости каналов управления, основанные на анализе устойчивости и качества регулирования каналов управления с учетом перекрестных связей в частотной области} \label{sec:ch2/sec5}

Представим уравнения ОУ и регулятора (рисунок~\ref{fig:img-14}) в форме передаточных функций (\labelcref{eq:p2:11-}) и оценим перекрестные связи линеаризованной двух связной САУ 

\begin{equation}
\label{eq:p2:11-}
\begin{alignedat}{4}
\vartheta{}=W_{11}(p)u_1+W_{12}\left(p\right)u_2 ,\\
\psi{}=W_{21}(p)u_1+W_{22}\left(p\right)u_2 ,\\
u_1=R_{11}(p){\epsilon{}}_1,u_2=R_{22}(p){\epsilon{}}_2 ,\\
{\epsilon{}}_1={\phi{}}_1-\vartheta{},{\epsilon{}}_{12}={\phi{}}_2-\psi{} ,\\
\end{alignedat}
\end{equation}

\begin{figure}[ht]
	\centering
	\includegraphics[width=0.8\linewidth]{img-14} 
	\caption{Структурная схема САУ}
	\label{fig:img-14}
\end{figure}

Обозначения: 
\begin{itemize}
	\item $\varphi_1, \varphi_2$ – входные координаты,
	\item $u_1, u_2$ - управляющие напряжения,
	\item $\vartheta,\psi$ - выходные координаты.
\end{itemize}
 
\textbf{Оценка перекрестных связей.} \label{sec:ch2/sec5/s1}
Оценим прямые и перекрестные связи САУ по их частотным характеристикам (ЧХ). 
Разомкнем один из каналов управления (рисунок~\ref{fig:img-14}) и запишем передаточную функцию (ПФ) разомкнутой системы с учетом 2-ого замкнутого канала управления при $\varphi_2=0$ (рисунок~\ref{fig:img-14}). 
В работе \cite[]{Karpov} показано, что ПФ разомкнутой системы приводится к виду (\labelcref{eq:p2:12-}), который можно упростить  и оценить при слабых перекрестных связях ($\left| \delta(j\omega) \right| <<1$)  

\begin{equation}
\label{eq:p2:12-}
\begin{alignedat}{4}
W_{\textit{раз1}}\left(p\right)=
R_{11}\left(p\right)W_{11}\left(p\right)\left[1+\delta{}\left(p\right)W_2\left(p\right)\right]=\\
R_{11}\left(p\right)W_{11}\left(p\right){\Delta{}}_1\left(p\right)\cong{}\\
R_{11}\left(p\right)W_{11}\left(p\right) ,
\end{alignedat}
\end{equation}

где $W_{2}(p)=\frac{R_{22}(p)W_{22}(p)}{1+R_{22}(p)W_{22}(p)}$, 

$\delta{}(p)=\frac{W_{12}(p)W_{21}(p)}{W_{1}(p)W_{22}(p)}=\delta{}_{12}(p)\delta{}_{21}(p)$,

\begin{equation}
\label{eq:p2:14-1}
\begin{alignedat}{2}
{\Delta{}}_1(j\omega{})=[1+\delta{}(j\omega{})W_2(j\omega{})])=\left\vert{}{\Delta{}}_1(j\omega{})\right\vert{}e^{j{\phi{}}_1(\omega{})}
\end{alignedat}
\end{equation}

с оценками упрощения в частотной области по амплитудной и фазовой характеристикам: 
\begin{equation}
\label{eq:p2:13-}
\begin{alignedat}{2}
\Delta{}_1(\omega{})\ <\ 1+M_2\delta{}(\omega{})<\ 1+M_2\delta{},\\
\phi{}_1(\omega{})=arcsin(M_2\delta{}(\omega{}))<\ arcsin(M_2\delta{}\ )
\end{alignedat}
\end{equation}
где 
$\delta{}=max_{\omega{}\in{}\Omega{}}\left\vert{}\delta{}(j\omega{})\right\vert{},M_i=max_{\omega{}\in{}\Omega{}}W_i(\omega{}),(i=1,2)$ - показатели колебательности i-х каналов управления.

Если в соответствии с оценками (\labelcref{eq:p2:13-}) построить годограф
$W_{\textit{раз1}}\left(p\right)=R_{11}\left(p\right)W_{11}\left(p\right)$
и трубку вокруг него с радиусами 
${\epsilon{}}_1\left(\omega{}\right)=R_{11}\left(\omega{}\right)W_{11}\left(\omega{}\right)\delta{}\left(\omega{}\right)M_2$, 
то на основе критерия Найквиста можно судить об устойчивости системы (\labelcref{eq:p2:10-}) с учетом перекрестных связей и о влиянии перекрестных связей на ее устойчивость. Аналогичные выражения и заключения получены и для 2-го канала управления
\begin{equation}
\label{eq:p2:14-}
\begin{alignedat}{2}
W_{\textit{раз2}}\left(p\right)=R_{22}\left(p\right)W_{22}\left(p\right)\left[1+\delta{}\left(p\right)W_1\left(p\right)\right]=\\
R_{22}\left(p\right)W_{22}\left(p\right){\Delta{}}_2\left(p\right)\cong{}\\
R_{22}\left(p\right)W_{22}\left(p\right),\left(14\right)
\end{alignedat}
\end{equation}

где $W_{1}(p)=\dfrac{R_{11}(p)W_{11}(p)}{1+R_{11}(p)W_{11}(p)}$, 

$\delta{}(p)=\dfrac{W_{12}(p)W_{21}(p)}{W_{1}(p)W_{22}(p)}=\delta{}_{12}(p)\delta{}_{21}(p)$,

\begin{equation}
\label{eq:p2:15-}
\begin{alignedat}{2}
{\Delta{}}_2(j\omega{})=[1+\delta{}(j\omega{})W_1(j\omega{})])=\left\vert{}{\Delta{}}_2(j\omega{})\right\vert{}e^{j{\phi{}}_2(\omega{})}
\end{alignedat}
\end{equation}

\begin{equation}
\label{eq:p2:15-2}
\begin{alignedat}{2}
\Delta{}_2(\omega{})\ <\ 1+M_1\delta{}(\omega{})<\ 1+M_1\delta{},\\
\phi{}_2(\omega{})=arcsin(M_1\delta{}(\omega{}))<\ arcsin(M_1\delta{}\ )
\end{alignedat}
\end{equation}
\textbf{Оценка показателей колебательности.} \label{sec:ch2/sec5/s2}
Показатель колебательности является наиболее информативным частотным критерием оценки качества регулирования и устойчивости САУ. Пусть М1, М2 - показатели колебательности изолированных САУ (без учета перекрестных связей). С учетом сделанных выше обозначений в работе \cite[]{Karpov} получены оценки показателей колебательности каналов управления с учетом перекрестных связей

\begin{equation}
\label{eq:p2:16-}
\begin{alignedat}{2}
\overline{M_1}<M_{11}\frac{1+M_2\delta{}}{1-M_2M_1\delta{}}\\
\overline{M_2}<M_{22}\frac{1+M_1\delta{}}{1-M_1M_2\delta{}}
\end{alignedat}
\end{equation}
где 
$\delta{}={\delta{}}_{12}{\delta{}}_{21}$,
${\delta{}}_{12}=max_{\omega{}\in{}\Omega{}}{\delta{}}_{12}\left(\omega{}\right)$, 
${\delta{}}_{21}=max_{\omega{}\in{}\Omega{}}{\delta{}}_{21}\left(\omega{}\right)$,
${\delta{}}_{12}\left(\omega{}\right)=\frac{W_{12}\left(\omega{}\right)}{W_{11}\left(\omega{}\right)}$, 
${\delta{}}_{21}\left(\omega{}\right)=\frac{W_{21}\left(\omega{}\right)}{W_{22}\left(\omega{}\right)}$.

С учетом оценок (\labelcref{eq:p2:12-,eq:p2:16-}) сформулируем условия декомпозируемости для двухканальных систем.

\textbf{Определение.}
 Если  i-е изолированные каналы разомкнутых систем управления устойчивы и выполняются условия: 
\begin{enumerate}
	\item $\omega \in [0,\inf)$,
	\item $\delta{}_{12}\left(\omega{}\right)<1$, 
	$\delta{}_{21}\left(\omega{}\right)<1$
	${\delta{}}_i(\omega{})={\delta{}}_{12}(\omega{}){\delta{}}_{21}(\omega{})M_j<\delta^{\textit{доп}}(M^{\textit{доп}}, \varDelta^{\textit{доп}}, \varphi^{\textit{доп}}) (i=1,2; j=2,1)$
	\item годографы 
	$W_{\textit{разi}}(j\omega{})=W_{ii}(j\omega{})W_{ii}(j\omega{})$ с трубкой радиуса 
	${\epsilon{}}_1\left(\omega{}\right)$ не охватывают точку $(-1,j0)$,
	\item оценки показателей колебательности $\overline{M_1}<M^{\textit{доп}}$,
\end{enumerate}
то декомпозиция системы (\labelcref{eq:p2:11-}) приемлема в смысле устойчивости и качества регулирования САУ; где
$\delta^{\textit{доп}}(M^{\textit{доп}}, \varDelta^{\textit{доп}}, \varphi^{\textit{доп}})$
 – функциональная зависимость точности декомпозируемости каналов управления   от допустимых требований качества регулирования $M^{\textit{доп}}$ и запасов устойчивости $\varDelta^{\textit{доп}}, \varphi^{\textit{доп}}$ САУ. 
 
 
\section{Синтез регуляторов частотным методом} \label{sec:ch2/sec6}


Синтез регуляторов проводится с учетом управляющих воздействий (3) и действующих моментов в классе комбинированных астатических систем, исходя из условий устойчивости, точности и качества регулирования. Предварительно выбираются привода с учетом моментов нагрузки ($M_{\textit{н}}$), скоростей и ускорений и носителя \cite[]{Babaev38-a2-13}. Модуль частотной характеристики разомкнутых изолированных каналов управления синтезируется из условия 

\begin{equation}
\label{eq:p2:14-a2}
\begin{alignedat}{2}
lk
\end{alignedat}
\end{equation}

где $\epsilon_i \leq (0.03-0.1)$ – точности инвариантности к колебаниям носителя, $k_f$ – коэффициент передачи привода по моменту. Постоянные времени привода ($T_{\textit{П}}$), усилителя ($T_{\textit{У}}$), датчика ($T_{\textit{д}}$) определяется из условий устойчивости 
(при $T_{\textit{П}}>T_{\textit{У}}>T_{\textit{д}}$) 
и показателя колебательности (М) \cite[]{Bessekerski-a2-14}:	

\begin{equation}
\label{eq:p2:15-a2}
\begin{alignedat}{2}
kjh
\end{alignedat}
\end{equation}

Если условия (\labelcref{eq:p2:15-a2}) не выполняются, то далее синтез проводится путем построения желаемых логарифмических амплитудных и фазовых характеристик (ЛАХ-$L(\omega)$, ЛФХ-$\varphi(\omega)$), реализующих требования  устойчивости (критерий Найквиста) и качества регулирования (частотные свойства разомкнутой системы) \cite[]{Bessekerski-a2-14}:

\begin{itemize}
	\item в низкочастотной области  ЛАХ – $20 lg(\alpha_{\textit{вх}}(\omega_{\textit{н}}) / \varDelta\alpha)=L(\omega_{\textit{н}})<L(\omega)$, 
	$20 lg K_c < L(\omega)$, где 
	$\omega_{\textit{н}}$ – частоты колебаний носителя;
	\item в среднечастотной области  ЛАХ – на частоте среза ($\omega_{\textit{ср}}$) наклон ЛАХ должен быть -20 дб/дек в диапазоне частот: \\
	$\varDelta\omega = \omega_c ( 1 \pm 0.5(M + 1)/(M - 1) )$, где М – показатель колебательности;
	\item запасы устойчивости – 
	по фазе $\varDelta\varphi>45-60$ град. и 
	по амплитуде $\varDelta L>6$ дБ,
	которые определяют указанные требования к параметрам датчиков, усилителей и корректирующих звеньев, последние определяются по формулам:
	\begin{equation}
	\label{eq:p2:16-a2}
	\begin{alignedat}{2}
	kjh
	\end{alignedat}
	\end{equation}
\end{itemize}

Далее синтез доопределяется на основе исследования динамики САУ на пространственной компьютерной модели [9]  с учетом  нелинейностей регулятора и объекта управления.

\section{Разработка компьютерной имитационной модели} \label{sec:ch2/sec7}

\section{Разработка и исследование} \label{sec:ch2/sec8}

\textbf{Разработка и исследование масштабного динамического макета}

Необходимость создания и исследования динамического макета (ДМ) возникает при разработке принципиально новых и модернизируемых САУ БОЭП с целью проверки заложенных принципов управления и стабилиза-ции изображения,что позволяет сократить стоимость и время их разработки. 

Размеры (L)  и массы (m)  ОУ (его узлов и деталей), необходимых мощностей (Р) и моментов  (М) приводов, сил пружин амортизаторов (F) ДМ определяются, исходя из выбранных масштабов ($\mu_i$) динамического подобия [17]:

\textbf{Испытание опытного образца на стендах}
 в соответствии с методиками испытаний и требований ТЗ: (16-17). Если критерии (2) не выполняются, то переходим  на 4-е круги последовательных итерационных процедур: (17-25-21-8,…17), (17-25-26-1,…17). Результаты испытаний отражают в протоколе и делают заключение о необходимости доработок СВ и САУ и проведения повторных испытаний или допуске их  к испытаниям на борту. 

\textbf{Испытания на борту}
 в соответствии с методиками натурных испытаний и требований ТЗ: (18-19). Если критерии (2) не выполняются, то переходим  на 5-е круги итераций: (19-27-23-3,…19), (19-27-26-1,…19). Результаты испытаний и требования к техническим характеристикам СВ и САУ фиксируем в протоколе испытаний и делаем заключение о необходимости доработок СВ и САУ или допуске их  к дальнейшему производству.

Приведенная методика была апробирована при разработке ряда САУ ОЭП [2,3,9-12]. Каждому из блоков на рис.1 присущи своя специфика и его математическое или логическое описание и предполагается соответствующая методика его реализации. Сущность их раскроем ниже.


\section{Выводы по главе} \label{sec:ch2/sec9}

Предлагаемая методика разработки и исследования, объединяя теорию оптического изображения, теорию автоматического управления и законы механики, методы математического и компьютерного моделирования в единое целое, позволяет решать важные прикладные задачи построения адекватных математических моделей бортовых ОЭП как объектов управления. На их основе с применением компьютерных технологий наиболее эффективно можно решать задачи синтеза САУ и СВ бортовых ОЭП, что позволит уменьшить сроки и стоимость их разработки. 
