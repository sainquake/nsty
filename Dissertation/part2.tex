\chapter{Методика разработки и исследования динамики систем управления бортовыми оптико-электронными приборами с применением информационных технологий} \label{ch:ch2}

В настоящее время наиболее распространены в народном хозяйстве и военной технике комплексированные оптико - электронные системы (КОЭП) визуализации, включающие в себя каналы наблюдения и зондирования в широком спектре волн оптического диапазона \cite[]{Tarasov},\cite[]{Belyakov},\cite[]{Karpov},\cite[]{Torshina}. Появление на рынке матричных фотоприемников расширило возможности КОЭП и К(комплексов). Несмотря на интенсивное развитие   теории и методов расчета и совершенствование бортовых автоматических КОЭП и К  возникают ряд вопросов, влияющих на качество получаемой оптической информации, в частности – это вопросы динамики и качества управления и увязки их с оптическими характеристиками каналов визуализации \cite[]{Belyakov},\cite[]{Karpov}, \cite[]{Baloev16}, \cite[]{Karpov17}.

Рассматриваются алгоритмы методики разработки и исследования динамики систем виброзащиты и управления бортовыми оптико- электронными приборами в виде десяти последовательных интерактивных замкнутых процедур от анализа технического задания до испытаний на борту.

Оптико-электронные приборы (\hyperref[acroEOS]{ОЭП}) авиационного, морского, наземного и космического базирования нашли широкое применение при выполнении задач наблюдения и охраны, в том числе при решении народнохозяйственных задач и задач обороны и безопасности.

В настоящее время широко рекламируются комплексированные \hyperref[acroEOS]{ОЭП} визуализации, включающие в себя каналы наблюдения и зондирования в широком спектре волн оптического диапазона, для применения в народном хозяйстве и военной технике \cite[]{Tarasov},\cite[]{Belyakov}, \cite[]{Torshina}, \cite[]{Ivanov18}. Появление на рынке матричных фотоприемников расширило возможности \hyperref[acroEOS]{ОЭП} и комплексов. Несмотря на интенсивное развитие   теории и методов расчета и совершенствование бортовых автоматических \hyperref[acroEOS]{ОЭП} возникает ряд вопросов, влияющих на качество получаемой оптической информации, в частности – это вопросы динамики и качества управления и увязки их с характеристиками каналов визуализации \cite[]{Tarasov},\cite[]{Belyakov}, \cite[]{Baloev16}, \cite[]{Karpov17}, \cite[]{Gerasin19}. Ниже в развитие работы \cite[]{Tarasov} рассматривается методика разработки и исследования систем автоматического управления (\hyperref[acroSAU]{САУ}) и виброзащиты (СВ) \hyperref[acroEOS]{ОЭП} с использованием замкнутых процедур исследования от разработки математических моделей до испытаний на борту носителя.


\section{Методика разработки и исследования динамики} \label{sec:ch2/sec1-}


Разработка СВ и \hyperref[acroSAU]{САУ} \hyperref[acroEOS]{ОЭП} начинается с выбора приемлемого варианта. Для этого решается много критерийная задача оптимизации с учетом ряда противоречивых технико-экономических (Т-Э) требований: точность \hyperref[acroSAU]{САУ} -  $\alpha_{1}$, полосы пропускания \hyperref[acroSAU]{САУ} и СВ - $\alpha_{21}$, $\alpha_{22}$; качество изображения -  $\alpha_{3}$, время экспозиции -  $\alpha_{4}$ , потребляемая энергия -  $\alpha_{5}$, надежность -  $\alpha_{6}$, масса объекта управления (ОУ) -  $\alpha_{7}$, стоимость -  $\alpha_{8}$, конкурентоспособность - $\alpha_{9}$, характеристики объектива -$\alpha_{10}$ и приемника излучения - $\alpha_{11}$ и т.п., которые определяются из технического задания (ТЗ) на \hyperref[acroEOS]{ОЭП} методом экспертных оценок специалистов в этих областях науки и техники с учетом предварительных расчетов и исследований. Критерий выбора приемлемого (i) варианта определяется по формуле:

\begin{equation}
\label{eq:p2:1}
\begin{alignedat}{2}
k_i=min\sum_{j=1}^n{\gamma _{ji}\alpha _{ji}}
\end{alignedat}
\end{equation}

где \textit{i} – число вариантов, \textit{n} – число Т-Э параметров, $\gamma_{ij}$ – весовые коэффициенты, $\alpha_{ji}$ – Т-Э параметры. Для выбранных приемлемых вариантов  СВ и \hyperref[acroSAU]{САУ} (одного или двух) в соответствии с методикой изложенной в главе \ref{ch:ch4} проводится исследование их динамики. За критерий качества \hyperref[acroSAU]{САУ} обычно принимают совокупность динамических характеристик каналов управления, удовлетворяющих условиям:

\begin{equation}
\label{eq:p2:2}
\begin{alignedat}{2}
\varDelta \alpha _k\leqslant \varDelta \alpha _{k}^{\textit{доп}},
\\
\,\,\,\,\varDelta \dot{\alpha}_k\leqslant \varDelta \dot{\alpha}_{k}^{\textit{доп}},
\\
\,\,\,\,\,\,M\leqslant \text{1.05..1.25,}
\\
\,\,\,\,\left| \left. \varDelta \varphi \right| \right. \geqslant \left( 45-60 \right) ^0,
\\
\,\,\,\,\left| \left. \varDelta L \right|\geqslant \,\,\textit{6дб}, \right. 
\end{alignedat}
\end{equation}

где  $\varDelta \alpha^{\textit{доп}}_{k}$, 
$\varDelta \dot{\alpha}^{\textit{доп}}_{k}$, 
$\varDelta \alpha _k$, 
$\varDelta \dot{\alpha}_k$ 
- допустимые установившиеся значения динамических погрешностей \hyperref[acroSAU]{САУ} и СВ по углу  и угловой скорости  при действии возмущений, полученных в условиях, близких к реальной эксплуатации \hyperref[acroSAU]{САУ}, \textit{k= 1,2,...} - номер канала управления, обеспечивающего качество изображения, М – показатель колебательности, $\varDelta \varphi$, $\varDelta L$ –запасы устойчивости по фазе и по амплитуде, полученные из логарифмической частотной характеристики  разомкнутой системы \cite[]{Bessekerski20}.

Согласно предлагаемой методике алгоритм разработки СВ, \hyperref[acroSAU]{САУ} и исследования их динамики представлен в виде 4-х основных последовательных интерактивных замкнутых процедур с использованием 27-ми блоков разработки и исследования (рисунок~\ref{fig:tikz_example}):

\begin{enumerate}
	\item После выполнения последовательных процедур (верификации, разработки расчетной и математической моделей, декомпозиции) с применением компьютерных технологий (Solid Works, MathCAD, MATLAB): (рисунок~\ref{fig:tikz_example}: 1,2,...7) проводится интерактивный синтез алгоритмов управления изолированных каналов \hyperref[acroSAU]{САУ} частотным методом \cite[]{Bessekerski20} и конструктивных параметров СВ (рисунок~\ref{fig:tikz_example}: 8-9-21-8) до выполнения критериев качества (глава \ref{ch:ch4}). На основании полученной информации проводится разработка и изготовление масштабного динамического макета \hyperref[acroEOS]{ОЭП} (ОУ, приводов, встроенных датчиков с обеспечением адекватности их динамическим характеристикам) с применением 3Д - принтера (рисунок~\ref{fig:tikz_example}: 10). Далее проводится исследование динамики пространственного движения макета \hyperref[acroSAU]{САУ} и СВ с использованием синтезированных алгоритмов управления (рисунок~\ref{fig:tikz_example}: 11-12). Если требования ТЗ (рисунок~\ref{fig:tikz_example}: 2) не выполняются, то проводятся последовательно циклы итерационных исследований динамики \hyperref[acroSAU]{САУ} и СВ : (рисунок~\ref{fig:tikz_example}: 12-20-10,...12), (рисунок~\ref{fig:tikz_example}: 12-20-21-8,...12), (рисунок~\ref{fig:tikz_example}: 12-20-23-3,...12) до обеспечения критериев качества (глава \ref{ch:ch4}), заключающихся в оптимальном выборе параметров регулятора и ОУ.
	\item Далее переходим к исследованию пространственной модели с применением MATLAB: 
	(рисунок~\ref{fig:tikz_example}: 13 – 14 - 15). Если критерии (глава \ref{ch:ch4}) не выполняются, то переходим на 2-е круги последовательных итераций: (рисунок~\ref{fig:tikz_example}: 15-24-21-8,...15), (рисунок~\ref{fig:tikz_example}: 15-24-23-3,...15), где доопределяем число степеней свободы математической модели и её параметры, параметры \hyperref[acroSAU]{САУ} и СВ путем последовательных предыдущих итераций исследований до выполнения критериев (глава \ref{ch:ch4}). C учетом полученной информации доопределяем: необходимые конструктивные доработки  ОУ, \hyperref[acroSAU]{САУ}, СВ ; приемлемые варианты построения СВ и \hyperref[acroSAU]{САУ}, а также в случае необходимости уточняем критерии (глава \ref{ch:ch4}) и задачи, которые могут решать СВ и \hyperref[acroSAU]{САУ}, а также ограничения и нелинейности в контурах управления. По результатам исследования делается заключение о необходимости изготовления опытного образца \hyperref[acroSAU]{САУ} и СВ или проведения дальнейших исследований.
	
	\item Затем переходим к испытаниям опытного образца \hyperref[acroSAU]{САУ} и СВ на стендах в соответствии с методиками испытаний и требований ТЗ: (рисунок~\ref{fig:tikz_example}: 16-17). Если критерии (глава \ref{ch:ch4}) не выполняются, то переходим  на 
	3-и круги последовательных итерационных процедур: (рисунок~\ref{fig:tikz_example}: 17-25-21-8,...17), (рисунок~\ref{fig:tikz_example}: 17-25-26-1,...17), которые включает в себя две предыдущих. Результаты исследований фиксируют в протоколе испытаний и делают заключение о необходимости доработок СВ и \hyperref[acroSAU]{САУ} или допуске их к испытаниям на борту. 
	
	\item В заключении переходим к испытаниям \hyperref[acroSAU]{САУ} и СВ на борту в соответствии с методиками натурных испытаний и требований ТЗ: (рисунок~\ref{fig:tikz_example}: 18-19). Если критерии (глава \ref{ch:ch4}) не выполняются, то переходим  на 
	4-и круги последовательных итерационных процедур: (рисунок~\ref{fig:tikz_example}: 19-27-23-3,...19), (рисунок~\ref{fig:tikz_example}: 19-27-26-1,...19), которые включают в себя (при необходимости) три предыдущие. Результаты испытаний и требования к техническим характеристикам СВ и \hyperref[acroSAU]{САУ} фиксируем в протоколе испытаний и делаем заключение о необходимости доработок СВ и \hyperref[acroSAU]{САУ} или допуске их к дальнейшему производству.
	Приведенная методика была апробирована при разработке ряда \hyperref[acroSAU]{САУ} \hyperref[acroEOS]{ОЭП} \cite[]{Belyakov}, \cite[]{Karpov}, \cite[]{Baloev16}, \cite[]{Karpov17}, \cite[]{Gerasin19}, \cite[]{Molin21}. Каждому из блоков на рисунке ниже (рисунок~\ref{fig:tikz_example}) присущи своя специфика и его математическое или логическое описание и предполагается соответствующая методика его разработки. Некоторые из них приводятся ниже.

\end{enumerate}

\section{Оценка допуска на точность стабилизации изображения} \cite[]{Belyakov}, \cite[]{Sokolski22}, \cite[]{Molin21} \label{sec:ch2/sec2} 



Для изучения процесса формирования изображения с учетом множества факторов, влияющих на формирование, преобразование и передачу качества изображения, рассмотрим функциональную схему одного из вариантов комплексированного \hyperref[acroEOS]{ОЭП} (рисунок~\ref{fig:oep_sch}), представляющего собой совокупность \hyperref[acroTVS]{тепловизионной системы (ТВС)}, \hyperref[acroTS]{телевизионной системы (ТС)}, \hyperref[acroFS]{фотографической системы (ФС)}, наблюдательных приборов в видимой и ближней инфракрасной (ИК) областях. В результате действия внешних возмущений $F(P,g,T,t)$, возмущений, идущих от носителя, и наличия управления \hyperref[acroEOS]{ОЭП} в пространстве остается не компенсированный сдвиг изображения (динамическая погрешность).


\begin{figure}[ht]
	{\centering
		\ifdefmacro{\tikzsetnextfilename}{\tikzsetnextfilename{tikz_example_compiled}}{}% присваиваемое предкомпилированному pdf имя файла
		% !TEX encoding = UTF-8 Unicode
% Úτƒ-8 encoded
% http://www.linux.org.ru/forum/general/10357036
\tikzset{
    line/.style={draw, -latex'},
    every join/.style={line},
    u/.style={anchor=south},
    r/.style={anchor=west},
    fxd/.style={text width = 6em},
    it/.style={font={\small\itshape}},
    bf/.style={font={\small\bfseries}}
}
\tikzstyle{base} =
    [
        draw,
        on chain,
        on grid,
        align=center,
        minimum height=4ex,
        minimum width = 10ex,
        node distance = 6mm and 60mm,
        text badly centered
    ]
\tikzstyle{coord} =
    [
        coordinate,
        on chain,
        on grid
    ]
\tikzstyle{cloud} =
    [
        base,
        ellipse,
        fill = red!5,
        node distance = 3cm,
        minimum height = 2em
    ]
\tikzstyle{decision} =
    [
        base,
        diamond,
        aspect=2,
        fill = green!10,
        node distance = 2cm,
        inner sep = 0pt
    ]
\tikzstyle{block} =
    [
        rectangle,
        base,
        fill = blue!3,
        rounded corners,
        minimum height = 2em
    ]
\tikzstyle{print_block} =
    [
        base,
        tape,
        tape bend top=none,
        fill = yellow!10
    ]
\tikzstyle{io} =
    [
        base,
        trapezium,
        trapezium left angle = 70,
        trapezium right angle = 110,
        fill = blue!5
    ]
\makeatletter
\pgfkeys{/pgf/.cd,
    subrtshape w/.initial=2mm,
    cycleshape w/.initial=2mm
}
\pgfdeclareshape{subrtshape}{
    \inheritsavedanchors[from=rectangle]
    \inheritanchorborder[from=rectangle]
    \inheritanchor[from=rectangle]{north}
    \inheritanchor[from=rectangle]{center}
    \inheritanchor[from=rectangle]{west}
    \inheritanchor[from=rectangle]{east}
    \inheritanchor[from=rectangle]{mid}
    \inheritanchor[from=rectangle]{base}
    \inheritanchor[from=rectangle]{south}
    \backgroundpath{
        \southwest \pgf@xa=\pgf@x \pgf@ya=\pgf@y
        \northeast \pgf@xb=\pgf@x \pgf@yb=\pgf@y
        \pgfmathsetlength\pgfutil@tempdima{\pgfkeysvalueof{/pgf/subrtshape w}}
        \def\ppd@offset{\pgfpoint{\pgfutil@tempdima}{0ex}}
        \def\ppd@offsetm{\pgfpoint{-\pgfutil@tempdima}{0ex}}
        \pgfpathmoveto{\pgfqpoint{\pgf@xa}{\pgf@ya}}
        \pgfpathlineto{\pgfqpoint{\pgf@xb}{\pgf@ya}}
        \pgfpathlineto{\pgfqpoint{\pgf@xb}{\pgf@yb}}
        \pgfpathlineto{\pgfqpoint{\pgf@xa}{\pgf@yb}}
        \pgfpathclose
        \pgfpathmoveto{\pgfpointadd{\pgfpoint{\pgf@xa}{\pgf@yb}}{\ppd@offsetm}}
        \pgfpathlineto{\pgfpointadd{\pgfpoint{\pgf@xa}{\pgf@ya}}{\ppd@offsetm}}
        \pgfpathlineto{\pgfpointadd{\pgfpoint{\pgf@xb}{\pgf@ya}}{\ppd@offset}}
        \pgfpathlineto{\pgfpointadd{\pgfpoint{\pgf@xb}{\pgf@yb}}{\ppd@offset}}
        \pgfpathclose
    }
}
\pgfdeclareshape{cyclebegshape}{
    \inheritsavedanchors[from=rectangle]
    \inheritanchorborder[from=rectangle]
    \inheritanchor[from=rectangle]{north}
    \inheritanchor[from=rectangle]{center}
    \inheritanchor[from=rectangle]{west}
    \inheritanchor[from=rectangle]{east}
    \inheritanchor[from=rectangle]{mid}
    \inheritanchor[from=rectangle]{base}
    \inheritanchor[from=rectangle]{south}
    \backgroundpath{
        \southwest \pgf@xa=\pgf@x \pgf@ya=\pgf@y
        \northeast \pgf@xb=\pgf@x \pgf@yb=\pgf@y
        \pgfmathsetlength\pgfutil@tempdima{\pgfkeysvalueof{/pgf/cycleshape w}}
        \pgfpathmoveto{\pgfqpoint{\pgf@xa}{\pgf@ya}}
\pgfpathlineto{\pgfpointadd{\pgfpoint{\pgf@xa}{\pgf@yb}}{\pgfpoint{0ex}{-\pgfutil@tempdima}}}
\pgfpathlineto{\pgfpointadd{\pgfpoint{\pgf@xa}{\pgf@yb}}{\pgfpoint{\pgfutil@tempdima}{0ex}}}
\pgfpathlineto{\pgfpointadd{\pgfpoint{\pgf@xb}{\pgf@yb}}{\pgfpoint{-\pgfutil@tempdima}{0ex}}}
\pgfpathlineto{\pgfpointadd{\pgfpoint{\pgf@xb}{\pgf@yb}}{\pgfpoint{0ex}{-\pgfutil@tempdima}}}
\pgfpathlineto{\pgfqpoint{\pgf@xb}{\pgf@ya}}
        \pgfpathclose
    }
}
\pgfdeclareshape{cycleendshape}{
    \inheritsavedanchors[from=rectangle]
    \inheritanchorborder[from=rectangle]
    \inheritanchor[from=rectangle]{north}
    \inheritanchor[from=rectangle]{center}
    \inheritanchor[from=rectangle]{west}
    \inheritanchor[from=rectangle]{east}
    \inheritanchor[from=rectangle]{mid}
    \inheritanchor[from=rectangle]{base}
    \inheritanchor[from=rectangle]{south}
    \backgroundpath{
        \southwest \pgf@xa=\pgf@x \pgf@ya=\pgf@y
        \northeast \pgf@xb=\pgf@x \pgf@yb=\pgf@y
        \pgfmathsetlength\pgfutil@tempdima{\pgfkeysvalueof{/pgf/cycleshape w}}
        \pgfpathmoveto{\pgfqpoint{\pgf@xb}{\pgf@yb}}
\pgfpathlineto{\pgfpointadd{\pgfpoint{\pgf@xb}{\pgf@ya}}{\pgfpoint{0ex}{\pgfutil@tempdima}}}
\pgfpathlineto{\pgfpointadd{\pgfpoint{\pgf@xb}{\pgf@ya}}{\pgfpoint{-\pgfutil@tempdima}{0ex}}}
\pgfpathlineto{\pgfpointadd{\pgfpoint{\pgf@xa}{\pgf@ya}}{\pgfpoint{\pgfutil@tempdima}{0ex}}}
\pgfpathlineto{\pgfpointadd{\pgfpoint{\pgf@xa}{\pgf@ya}}{\pgfpoint{0ex}{\pgfutil@tempdima}}}
\pgfpathlineto{\pgfqpoint{\pgf@xa}{\pgf@yb}}
        \pgfpathclose
    }
}
\makeatother
\tikzstyle{subroutine} =
    [
        base,
        subrtshape,
        fill = green!25
    ]
\tikzstyle{cyclebegin} =
    [
        base,
        cyclebegshape,
        fill = blue!25
    ]
\tikzstyle{cycleend} =
    [
        base,
        cycleendshape,
        fill = blue!25
    ]
\tikzstyle{connector} =
    [
        base,
        circle,
        fill = red!25
    ]

\begin{tikzpicture}[%
    start chain=going below,    % General flow is top-to-bottom
    node distance=6mm and 150mm, % Global setup of box spacing
        ]
        \node [cloud] (start) {Начало};
        \node [block, join] (phase1) {1,2};
        \node [block, join] (phase3) {3};
        \node [block, join] (phase4) {4567};
        \node [block, join] (phase8) {8};
        \node [block, join] (phase9) {9};
        \node [block, join] (phase10) {10};
        \node [block, join] (phase11) {11};
        \node [decision, join, right of = phase11, node distance = 4cm] (condition12) {12};
        \node [block, join, left of = condition12, node distance = 4cm] (phase13) {13,14};
        \node [decision, join, right of = phase13, node distance = 4cm] (condition15) {15};
        \node [block, join, left of = condition15, node distance = 4cm] (phase16) {16};
        \node [decision, join, right of = phase16, node distance = 4cm] (condition17) {17};
        \node [block, join, left of = condition17, node distance = 4cm] (phase18) {18};
        \node [decision, right of = phase18, node distance = 4cm] (condition19) {19};
        \node [cloud] (fin) {Конец};
        
        \node [block, right of = phase1, node distance = 4cm] (print26) {26};
        \node [block, right of = phase3, node distance = 4cm] (print23) {23};
        \node [block, right of = phase8, node distance = 4cm] (print21) {21};
        \node [decision, right of = condition12, node distance = 4cm] (print20) {20};
        \node [decision, right of = condition15, node distance = 4cm] (print24) {24};
        \node [decision, right of = condition17, node distance = 4cm] (print25) {25};
        \node [decision, right of = condition19, node distance = 4cm] (print27) {27};
        
        \path [line, red] (condition12) -| node [u,near start] {Нет} (print20);
        \path [line, red] (condition15) -| node [u,near start] {Нет} (print24);
        \path [line, red] (condition17) -| node [u,near start] {Нет} (print25);
        \path [line, red] (condition19) -| node [u,near start] {Нет} (print27);
        
        \path [line, green] (condition19) to node [r] {Да}(fin);

\end{tikzpicture}

		
	}
	\legend{}
	\caption[Пример \texttt{tikz} схемы]{Пример рисунка, рассчитываемого
		\texttt{tikz}, который может быть предкомпилирован}
	\label{fig:tikz_example}
\end{figure}


Это приводит к изменению структуры изображения аналогично влиянию аберраций, т.е. к ухудшению качества изображения, зависящего от величины и характера изменения динамических погрешностей управляющих систем. Для оценки допуска на точность стабилизации изображения используют частотный критерий качества изображения \hyperref[acroEOS]{ОЭП} – функцию передачи модуляции (\hyperref[acroFPM]{ФПМ}) \cite[]{Tarasov}.

\begin{figure}[ht]
	\centering
	\includegraphics[width=0.8\linewidth]{oep_sch} 
	\caption{Функциональная схема обобщённой модели \hyperref[acroEOS]{ОЭП}}
	\label{fig:oep_sch}
\end{figure}

Обозначения: 
\begin{itemize}
	\item $F(P,g,T,t)$ – функция возмущений, где
	\begin{itemize}
		\item $P$ – атмосферное давление,
		\item $g$ – земное ускорение,
		\item $T$ - температура, 
		\item $t$ – время;
	\end{itemize}
	\item ФП – фотоприемник, 
	\item ЭОП – электронно-оптический преобразователь, 
	\item ССк – система сканирования, 
	\item У-П – усилитель-преобразователь, 
	\item СКИ – система коррекции изображения.
\end{itemize}

Оценка качества изображения управляемых \hyperref[acroEOS]{ОЭП} определяется на основе анализа произведения \hyperref[acroFPM]{ФПМ} элементов оптико-электронного тракта, формирующих изображение и влияющих на качество изображения. В рамках принятой структуре \hyperref[acroEOS]{ОЭП} его \hyperref[acroFPM]{ФПМ} должна удовлетворять по каждому каналу визуализации условию \cite[]{Ivanov18}, \cite[]{Molin21}.

\begin{equation}
\label{eq:p2:2.5}
\begin{alignedat}{2}
T_{\textit{ОЭП}}(N)=
T_{\textit{ат}}(N)T_{\textit{об}}(N)T_{\textit{фп}}(N)T_{\textit{пои}}(N)T_{\textit{САУ}}(N)>
T_{\textit{ОЭC}}^{\textit{доп}}(N)
\end{alignedat}
\end{equation}

\begin{equation}
\label{eq:p2:3}
\begin{alignedat}{2}
T_{\textit{ОЭC}}^{\textit{доп}}\left(\nu{}\right)=
\frac{N{\gamma{}}_\textit{и}}{2\sin{\left(N\frac{{\gamma{}}_\textit{и}}{2}\right)}}
\exp{\left[-\frac{{\left(N{\gamma{}}_p\right)}^2}{16\ln{\left(2m\right)}}\right]}
\end{alignedat}
\end{equation}

\begin{equation}
\label{eq:p2:4}
\begin{alignedat}{2}
T_{\Sigma{}}^{\textit{доп}}\left(N\right)=\exp{\left(-2{\pi{}}^2{\sigma{}}^2N^2\right)}
\end{alignedat}
\end{equation}

где 
$Т_{\textit{оэп}}(N)$ – \hyperref[acroFPM]{ФПМ}  \hyperref[acroEOS]{ОЭП}, 
$N$ – пространственная частота, 
$m$ - отношение сигнал/шум, 
$\gamma_\textit{и}$ - угловой размер источника излучения, 
$\gamma_\textit{p}$ - угловое разрешение  \hyperref[acroEOS]{ОЭП}, 
$T_{\textit{ОЭП}}^{\textit{доп}}(N)$ – допустимая \hyperref[acroFPM]{ФПМ}  \hyperref[acroEOS]{ОЭП}, 
$Т_\textit{ат}(N)$ – \hyperref[acroFPM]{ФПМ} атмосферы, 
$T_\textit{oб}(N)$ – \hyperref[acroFPM]{ФПМ} объектива, 
$Т_\textit{фп}(N)$ – \hyperref[acroFPM]{ФПМ} фотоприемника, 
$Т_\textit{пои}(N)$ – \hyperref[acroFPM]{ФПМ} преобразования оптической информации, вид которых можно найти  в \cite[]{Tarasov}, 
$T_{\textit{САУ}}(N)$ – \hyperref[acroFPM]{ФПМ} сдвига изображения (динамической погрешности \hyperref[acroSAU]{САУ}), зависящая от вида динамического смещения изображения: линейного (Л) - $\dot{x}(t) = Vt$, 
гармонического (Г) - $x(t)=a_{0}\sin{\omega t}$ и 
случайного (СЛ) \cite[]{Tarasov}, \cite[]{Sokolski22}.

Тогда в частном случае допустимая \hyperref[acroFPM]{ФПМ} \hyperref[acroSAU]{САУ} (системы слежения (Л, Г), системы виброзащиты (СВ), автоматической фокусировки (САФ)) определится соответственно: 

\begin{equation}
\label{eq:p2:6}
\begin{alignedat}{2}
T_{\textit{САУ}}\left(N\right)=
T_\textit{Л}\left(N\right)T_\textit{Г}\left(N\right)T_{\textit{СВ}}\left(N\right)T_{\textit{САФ}}\left(N\right)\geq{}T_{\textit{САУ}}^{\textit{доп}}(N)= \\
\dfrac{ T_{\textit{оэп}}^{\textit{доп}}(N) }{ T_{\textit{ат}}(N)T_{\textit{об}}(N)T_{\textit{фп}}(N)T_{\textit{пои}}(N) }
\end{alignedat}
\end{equation}

\begin{equation}
\label{eq:p2:7}
\begin{alignedat}{3}
T_\textit{Л}\left(N\right)=Sinc(\pi{}VtN) , \\
T_{\textit{Г}}(N)=J_0\left(2\pi{}a_0N\right) , \\
T_{\textit{СВ}}\left(N\right) = exp[-2{(\pi{}a_{\textit{ср}}N)}^2]
\end{alignedat}
\end{equation}

где $T_{\textit{САФ}}(N)$ – ФПМ САФ; 
$V$ – допустимая скорость движения изображения; 
$J0()$ - функция Бесселя нулевого порядка, 
$a_0$ –допустимая амплитуда гармонических колебаний,   
$a_{\textit{ср}}$ – среднее значение допустимой амплитуды случайного сдвига изображения. 
Выражения для оценки допустимых динамических погрешностей для ССАУ и СВ можно найти в \cite[]{Karpov}, \cite[]{Karpov23}. Допуск на точность САФ для фотообъективов в видимой области спектра, инфракрасных систем можно найти в \cite[]{Tarasov}, \cite[]{Belyakov}.

Пространственная расчетная частота Np должна соответствовать средним пространственным частотам (частоте Найквиста ($N_{\textit{н}} (\textit{мм}^{-1})$ либо $f_H (\textit{рад}^{-1})$), которая выбирается c учетом условий \cite[]{Tarasov}: 

\begin{equation}
\label{eq:p2:8}
\begin{alignedat}{2}
N_H=0.5 N_{\textit{выб}}\leq{}N_{\textit{гр}},
N_{\textit{гр}}=\frac{D}{1.22\lambda{}f^{'}},\\
f_H=\frac{X_\textit{э}}{2f^{'}},
f_{\textit{гр}}=\frac{D}{\lambda{}}(\textit{рад}^{-1}),
f_{\textit{выб}}=\frac{n_{\textit{э}}}{2\omega{}},\\
D\geq{}\frac{1.22 k_{\textit{аб}}\lambda{}}{h_{\textit{кр}}},
N_x=\frac{N_{\textit{д}}l}{h_{\textit{кр}}f^{'}},\\
f_x=\frac{N_{\textit{д}}l}{h_{\textit{кр}}},
f_{xy}=\frac{N_{\textit{д}}l}{\sqrt{h_{\textit{крx}}h_{\textit{крy}}}}=\sqrt{f_x^2+f_y^2},\\
\Delta{}{\omega{}}_p=\frac{2h_{\textit{кр}}}{N_{\textit{д}}l},
\Delta{}{\omega{}}_{\textit{аб}} =\frac{2.44 k_{\textit{аб}}\lambda{}}{D},
\Delta{}{\omega{}}_{\textit{э}} =\frac{d_{\textit{э}}}{f^{'}},\\
\frac{D}{f^{'}}=\frac{2.44 k_{\textit{аб}}\lambda{}}{d_{\textit{э}}},\\
\end{alignedat}
\end{equation}

где 
$N_{\textit{гр}}$ – граничная пространственная частота, 
$f_{\textit{выб}}$ – частота выборки, 
$X_{\textit{э}} ,n_{\textit{э}}$ –период расположения и число элементов фотоприемника, 
$2\omega$ - поле зрения ОЭП, 
$k_{\textit{аб}}$ – коэффициент, учитывающий аберрации объектива, 
$h_{\textit{кр}}$ – критический размер объекта наблюдения (ОН), 
$l$ – расстояние до ОН, 
$N_{x} , f_{x}$ – пространственные частоты, соответствующие критериям Джонсона, 
$N_{\textit{д}}$ – числа элементов разрешения (критерии Джонсона – обнаружения, классификации, распознавания и идентификации), 
$f_{xy}$ - пространственная частота Джонсона в двух ортогональных направлениях \textit{x} и \textit{y}, 
$\Delta{}{\omega{}}_p$ – требуемое геометрическое угловое разрешение, необходимое для наблюдения, 
$\Delta{}{\omega{}}_{\textit{аб}}$ – минимальное угловое значение кружка рассеяния объектива с учетом аберраций, 
$\Delta{}{\omega{}}_{\textit{э}} , d_{\textit{э}}$ – угловой и линейный размеры элемента чувствительного слоя приемника излучения. 

Если при $N \ge N_p$ качество изображения ОЭС будет удовлетворять условию $Т(N) \ge 0.8$, то оно считается хорошим. Таким образом, используя  (\labelcref{eq:p2:6,eq:p2:7,eq:p2:8}), можно определить допустимые динамические погрешности, обеспечивающие выполнение условия (\ref{eq:p2:6}).

\section{Разработка математической модели} \label{sec:ch2/sec3}

При разработке ОЭП одной из важнейших задач является управление направлением линии визирования в пространстве. Оно осуществляется, как правило, двумя способами: путем управления всем устройством информационных каналов (ОЭП 1-го типа) и управлением положением отдельных оптических элементов (зеркал, призм) (ОЭП 2-го типа) \cite[]{Karpov23}. Оба способа построения управляющих ОЭП имеют свои особенности и широко применяются.

Для построения математических моделей относительного движения ОЭП (движения относительно корпуса летательного аппарата (ЛА), на котором они устанавливаются) используются уравнения Лагранжа II-го рода. За инерциальную систему отсчета принимается система координат, связанная с поверхностью Земли. Для записи уравнений Лагранжа II-го рода используется смешанный метод Жильбера \cite[]{Belyakov}, \cite[]{Baloev16} [13]. Для изучения динамики ОЭП 1-го типа моделируется двумя абсолютно твердыми телами: «вилка» - азимутальный блок, в которой установлено второе тело - оптико-электронный блок (ОЭБ). Оси вращения «вилки» и ОЭБ взаимно перпендикулярны и пересекаются. Угол поворота «вилки» по азимуту – $\alpha$, угол поворота ОЭБ по углу места –$\beta$. Математическая модель относительного движения ОЭП 1-го типа определяется следующим матричным уравнением \cite[]{Karpov23}:

\begin{equation}
\label{eq:p2:6-}
\begin{alignedat}{2}
A_c\left(q_c\right){\ddot{q}}_c+B_c\left(t,q_c\right){\dot{q}}_c+Q_c\left(t,q_c\right)+\\
F_c\left(t,q_c\right)+F1_c\left(q_c,{\dot{q}}_c\right)=\frac{c_M}{r}u-M_{\textit{тр}}
\end{alignedat}
\end{equation}

где
$q_c=\left(\begin{array}{cc}
\alpha{} \\
\beta{}
\end{array}\right)$,      
$u=\left(\begin{array}{
	cc}
u_1 \\
u_2
\end{array}\right)$

$M_{\textit{тр}}=\left(\begin{array}{
		cc}
	M_{\textit{тр.1}}sign\left(\dot{\alpha{}}\right) \\
	M_{\textit{тр.2}}sign\left(\dot{\beta{}}\right)
\end{array}\right)$, 
$A_c\left(q_c\right),B_c\left(t,q_c\right)$ – матрицы-функции  размерности $2x2$,
$Q_c\left(t,q_c\right),F_c\left(t,q_c\right),F1_c\left(q_c,{\dot{q}}_c\right)$ – столбцы-функции размерности $2x1$,
$u_1$, $u_2$ - управляющее сигналы на моментных двигателях по азимуту и углу места. 

Азимутальный блок и угломестный блок ОЭП 2-го типа совместно с соосными приводами моделируется двумя подвижными твердыми телами, установленными в кардановом подвесе прибора, который закреплен на ЛА. Положение этих тел относительно корпуса однозначно определяется углами поворотов азимутального ($\phi_1$) и угломестного ($\phi_2$) блоков. Математическая модель относительного движения ОЭП 2-го типа определяется следующим матричным уравнением \cite[]{Baloev16}:

\begin{equation}
\label{eq:p2:7-}
\begin{alignedat}{2}
A\left(\phi{}\right)\ddot{\phi{}}+N\left(\phi{},\dot{\phi{}}\right)+H\left(\phi{},{\omega{}}_1\right)\dot{\phi{}}+\\
L\left(\phi{}\right){\epsilon{}}_1(t)-\Omega{}\left(\phi{},{\omega{}}_1\right)=\frac{c_M}{r}u-M_{mp}-P\left(\phi{},a\right)
\end{alignedat}
\end{equation}

$\phi{}=\left(\begin{array}{
	cc}
{\phi{}}_1 \\
{\phi{}}_2
\end{array}\right)$,     $u=\left(\begin{array}{
	cc}
u_1 \\
u_2
\end{array}\right),$   $M_{mp}=\left(\begin{array}{
	cc}
M_{\textit{mp.1}} \dot{\phi_1}   \\
M_{\textit{mp.2}} \dot{\phi_2}
\end{array}\right)$,

$A\left(\phi{}\right),H\left(\phi{},{\omega{}}_1\right)$ – матрицы-функции  размерности 2x2, $L(\phi)$– матрица-функция размерности 2x3,
$P\left(\phi{},a\right),N\left(\phi{},\dot{\phi{}}\right),\Omega{}\left(\phi{},{\omega{}}_1\right)$ –столбцы-функции размерности 2×1,
${\omega{}}_1(t)=\left(\begin{array}{
	ccc}
{\omega{}}_{X1}(t) \\
{\omega{}}_{Y1}(t) \\
{\omega{}}_{Z1}(t)
\end{array}\right)$
,    ${\epsilon{}}_1(t)=\left(\begin{array}{
	ccc}
{\dot{\omega{}}}_{X1}(t) \\
{\dot{\omega{}}}_{Y1}(t) \\
{\dot{\omega{}}}_{Z1}(t)
\end{array}\right)=\left(\begin{array}{
	ccc}
{\epsilon{}}_{X1}(t) \\
{\epsilon{}}_{Y1}(t) \\
{\epsilon{}}_{Z1}(t)
\end{array}\right)$
– векторы, составленные из проекций векторов угловой скорости и углового ускорения ЛА  на оси, связанные с корпусом ОЭП. Значения элементов матриц-функций определяется величинами осевых и центробежных моментов инерции, положением центров масс и массами тел прибора. В уравнениях (2.6),(2.7) зависимость от времени определяется уравнениями движения ЛА (1(t)).

\section{Верификация параметров} \label{sec:ch2/sec4}

\section{Оценки декомпозируемости каналов управления, основанные на анализе устойчивости и качества регулирования каналов управления с учетом перекрестных связей в частотной области} \label{sec:ch2/sec5}

\section{Синтез регуляторов частотным методом} \label{sec:ch2/sec6}

\section{Разработка КИМ и исследование динамики пространственной модели} \label{sec:ch2/sec7}

\section{Разработка алгоритмов управления БОЭП} \label{sec:ch2/sec8}

\section{Разработка компьютерной имитационной модели} \label{sec:ch2/sec9}

\section{Макетные испытания} \label{sec:ch2/sec10}

\section{Выводы по главе} \label{sec:ch2/sec11}



ll
