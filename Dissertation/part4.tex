\chapter{Разработка алгоритмов управления БОЭП} \label{ch:ch4}

Объектом исследования является системы автоматического управления (САУ) оптико-электронным прибором (ОЭП) в пространстве по азимуту и углу места в режимах наведения и стабилизации.

Цель работы – является разработка компьютерной имитационных моделей изолированных каналов управления ОЭП, управляемого моментными двигателями по азимуту и углу места и исследование его динамических свойств.
В процессе работы проводились выбор моментных двигателей, синтез алгоритмов управления, разработка компьютерных имитационных моделей изолированных нелинейных каналов управления бортового ОЭП и исследование динамики управления им в режимах наведения и стабилизации.

В результате исследования динамики предложены алгоритмы управления и требования к структуре и элементам изолированных каналов управления, обеспечивающие требования ТЗ, которые будут уточнены при исследовании пространственной задачи управления.

Эффективность управления в режиме наведения достигается применением оптимального управления по быстродействию. 
Все расширяющийся круг военных и гражданских задач, для решения которых широко используются оптико-электронные методы и средства, вызвало интерес к решению научно- технических   проблем, возникающих при решении этих задач \cite[]{Tarasov}, \cite[]{Fedoseev}.

Одна из важных проблем – это влияние динамики и управления на достижение требуемых тактико- технических характеристик бортовых ОЭП и комплексов. Вопросы разработки, моделирования и исследования динамических систем, в том числе бортовых оптико-электронных комплексов, отражены в трудах международных Четаевских конференций по аналитической механике, устойчивости и управлению; в трудах международных симпозиумов по автоматическому управлению и всесоюзных совещаниях по теории инвариантности, в трудах КАИ, а также в журналах: «Оптический журнал», «Гироскопия и навигация», «Оптико-механическая промышленность», «Автоматика и телемеханика», «Авиационная техника», «Вестник КГТУ им. А. Н. Туполева» [4-26]. Разработке алгоритмов и исследованию динамики, и управлению посвящены работы [3 - 29].

На данном этапе разработки выполняемой НИР проведены предварительные расчеты по уяснению решаемых задач и анализу исходных данных, оговоренных техническим заданием (ТЗ), разработке и согласованию
расчетных оптико – механической схем и параметров ОЭП, приводов, датчиков угла САУ с учетом их технических требований и внешних воздействий от носителя.


\section{Технические требования и режимы управления ОЭП} \label{ch:ch4/sect1}

В соответствии с требованиями ТЗ были проанализированы, уточнены, доопределены следующие исходные данные, необходимые для расчета:
\begin{itemize}
	\item Состав системы автоматического управления,
	\item САУ по азимуту,
	\item САУ по углу места,
	\item Датчики углового положения по азимуту и углу места,
	\item Двигатели типа ДБМ (уточняются в процессе разработки) по азимуту и углу места,
	\item Усилительно-преобразовательные блоки каналов управления совместно с микропроцессорными блоками,
	\item Назначение и технические требования.
\end{itemize}
САУ предназначена для наведения оси визирования на заданные угловые координаты относительно строительных осей летательного аппарата (ЛА), параметры движения ЛА.

\newpage
%%%%%%%%%%%%%%%%%%%% Table No: 9 starts here %%%%%%%%%%%%%%%%%%%%

\begin{table}[!h]
	\caption{Технические параметры САУ}%
	\label{tab:SAU_PARAM}% label всегда желательно идти после caption
	\begin{longtable}{|m{5cm}|m{5.5cm}|m{5.5cm}|}
		\hline
		& по азимуту & по углу места \\ 
\hline
Пределы наведения		& $\pm$ 180 град           & (-60 .. +30) град              \\ 
\hline
Максимальная скорость наведения		&600 град/сек            &300 град/сек               \\ 
		\hline
Максимальная среднеквадратическая погрешность наведения и стабилизации		&<30 угловых минут           &<30 угловых минут               \\ 
		\hline
Время наведения на предельный угол		&<0.6 с            & <0.6 с              \\ 
		\hline
Синусоидальные вибрации		& \multicolumn{2}{m{11cm}|}{
	\begin{tabular}[l]{m{11cm}}
		Частота 5-10 Гц, амплитуда перемещения 5 мм\\
		Частота 20-22 Гц, ускорение $25 \textit{м/с}^2$\\
		Частота 35,4-50 Гц, амплитуда перемещения 0.5 мм\\
		Частота 50-500 Гц, ускорение $25 \textit{м/с}^2$
	\end{tabular}

}      \\ 
		\hline
Требования к САУ при внешних воздействиях		& \multicolumn{2}{m{11cm}|}{\begin{tabular}[l]{m{11cm}}
		максимальной скорости движения носителя – 56 м/с (200км/час)\\
		максимальный крен ЛА $\leq 15^0$\\
		угловая скорость носителя $\omega_y=12$ град/c\\
		минимальная высота полета носителя – 1500 м\\
		максимальная скорость движения ОН – 500 м/с	
\end{tabular}}      \\ 
		\hline
Гармонические возмущения, идущие от ЛА по азимуту и углу места		& \multicolumn{2}{m{11cm}|}{\begin{tabular}[l]{m{11cm}}
		Частота 0.16 Гц, амплитуда колебаний 12 град.\\
		Частота 1 Гц, амплитуда колебаний 1- 2 град.
\end{tabular}}      \\ 
		\hline
Требования к САУ при воздействиях ускорений в условиях движения носителя		& \multicolumn{2}{m{11cm}|}{$4.8 \textit{м/с}^2$}      \\ 
		\hline
	\end{longtable}
\end{table}

%%%%%%%%%%%%%%%%%%%% Table No: 9 ends here %%%%%%%%%%%%%%%%%%%%

Геометрия масс тел вращения объекта управления (ОУ) по двум осям управления опредставлена в таблице \ref{tab:MASS/3.1} и рисунокe \ref{fig:41}.

%%%%%%%%%%%%%%%%%%%% Figure/Image No: 30 starts here %%%%%%%%%%%%%%%%%%%%

\begin{figure}[!ht]
	\centering
	\includegraphics[width=0.8\linewidth]{img-25} 
	\caption{Обозначение строительных осей ЛА}
	\label{fig:41}
\end{figure}

%%%%%%%%%%%%%%%%%%%% Figure/Image No: 30 Ends here %%%%%%%%%%%%%%%%%%%%

В отчете приведены синтез алгоритмов управления, разработка компьютерных моделей и исследование динамики изолированных каналов управления САУ с учетом частоты ШИМ и насыщения усилителей мощности, дискретизации датчиков угла.


\section{Обоснование выбора приводов} \label{ch:ch4/sect2}

Требуемая мощность и развиваемый момент двигателя определяются по формулам \cite[]{Bessekerski}:\par


\begin{equation}
	\label{eq:p4:1}
	\begin{multlined}
P \geq 2 \left( M_{\textit{нагр}}+I_{\textit{нагр}} \ddot \alpha^{max}   \right) \dot \alpha^{max},\\
M_{\textit{дв}} \geq 2 \left[ M_{\textit{нагр}}+I_{\textit{нагр}} \ddot \alpha^{max}  \right], \\
M_{\textit{нагр}}=M_{\textit{тр}}+M_{\textit{дб}}.
\end{multlined}
\end{equation}

Здесь 
$P$ - мощность двигателя, 
$М_{\textit{дв}}$ – момент на валу двигателя, 
$M_{\textit{нагр}}$ - статический момент нагрузки, 
$I_{\textit{нагр}}$ - момент  инерции нагрузки, 
$М_{\textit{тр}}$ - момент трения на валу двигателя,
$М_{\textit{дб}}$ - момент дисбаланса нагрузки,  
$\ddot \alpha^{max}, \dot \alpha^{max}$ - максимальные угловые скорость и ускорение.\par

Оценим указанные параметры двигателей. 
В соответствии с требованиями ТЗ следует, что наиболее энергоемким является режим наведения. Ставится задача перевода оптической оси ОЭП относительно строительных осей ЛА за время \textit{0.6 с}. из одного положения в другое в пределах углов: по азимуту – $2 \alpha_0 = 360$ град., по углу места – $2 \beta_0 = 90$ град.  \par

Для решения этой задачи рассмотрим оптимальное управление по быстродействию (по минимуму времени перевода оптической оси ОЭП из одной точки пространства в другую) \cite[]{Babaev},\cite[]{Boltanski}. Суть такого управления состоит в том, что для перевода оптической оси на требуемый угол за минимальное время необходимо в первой половине этого времени разгоняться с постоянным положительным ускорением, а в другой половине – тормозить с постоянным отрицательным ускорением. Таким образом изменение угла поворота ОЭП будет проходить по нелинейной траектории (двум параболам).\par

Требуемое угловые ускорение и скорость ротора моментного двигателя с нагрузкой по азимуту, необходимое для достижения необходимого половинного угла $2 \beta_0 = 45$ град. за время $\tau=0.3$ с:\par

\begin{equation}
\label{eq:p4:1+}
\begin{multlined}
  \alpha _{0} = \frac{ \ddot \alpha _{0} \tau^{2}}{2} \rightarrow 
\ddot \alpha _{0} = \frac{2 \alpha _{0}}{ \tau^{2}} = \frac{2 \pi /2}{  0.3^{2}} = 34.906 \textit{рад/с}^{2},\\
\dot \alpha ^{max}= \ddot \alpha _{0} \tau = 34.906\cdot 0.3=10.47 \textit{рад/с}. 
\end{multlined}
\end{equation}

Требуемое угловые ускорение и скорость ротора моментного двигателя с нагрузкой по углу места, необходимое для достижения необходимого половинного угла $\beta_0=45$ град. за время $\tau=0.3$ с:

\begin{equation}
\label{eq:p4:1+1}
\begin{multlined}
\beta _{0} = \frac{ \ddot \beta _{0} \tau^{2}}{2} \rightarrow 
\ddot \beta _{0} = \frac{2 \beta _{0}}{ \tau^{2}} = \frac{2 \pi /4}{  0.3^{2}} = 17.453 \textit{рад/с}^{2},\\
\dot \beta ^{max}= \ddot \beta _{0} \tau = 17.453\cdot 0.3=5.235 \textit{рад/с}. 
\end{multlined}
\end{equation}
Моменты трения оценим по формуле:

\begin{equation}%\tag{44}
\label{eq:p4:4}
\begin{multlined}
	M_{\textit{тр}}= \left( 2  \div 5 \right) m 10^{-3} \left[ \textit{Нм} \right] 
\end{multlined}
\end{equation}
где \textit{m} – масса подвижной части прибора. С учетом данных из таблицы \ref{tab:MASS/3.1} (m1=15.7 кг, m2=1.131 кг) находим по формуле 41):\par

\begin{equation}%\tag{44}
\label{eq:p4:4+}
\begin{multlined}
M_{\textit{ТР1}} = 5 \cdot 10^{-3}\cdot 15.7 = 0.0785 \textit{Нм},\\
M_{\textit{ТР2}} = 5 \cdot 10^{-3}\cdot 1.131 = 0.005755 \textit{Нм}
\end{multlined}
\end{equation}

Моменты дисбаланса нагрузки определим по формулам:
\begin{equation}%\tag{44}
\label{eq:p4:4+1}
\begin{multlined}
M_{\textit{дб1}} = 
m_{1} \left( a+g \right) \sqrt[]{x_{c1}^{2}+y_{c1}^{2}} =
15.7 \left( 4.8 + 9.8 \right) \cdot 9 \cdot 10^{-3}=2.06 \textit{Нм},\\
M_{\textit{дб2}}=
m_{2} \left( a+g \right) y_{c2}=
1.131 \cdot 14.8 \cdot 10^{-3} = 0.245 \textit{Нм},
\end{multlined}
\end{equation}

здесь \textit{а }– ускорение, создаваемое носителем, \textit{g} – ускорение земного притяжения, \textit{х\textsubscript{с1},у\textsubscript{с1},у\textsubscript{с2}} – величины дисбаланса ОЭП (таблица \ref{tab:MASS/3.1})\par

Максимальные статические моменты нагрузки на валу двигателей по азимуту и углу места соответственно равны:\par

\begin{equation}%\tag{44}
\label{eq:p4:4+2}
\begin{multlined}
M_{\textit{нагр1}}=M_{\textit{ТР1}}+M_{\textit{дб1}}=0.078+2.064=2.1425 \textit{Нм}, \\
M_{\textit{нагр2}}=0.005755+0.245=0.250755 \textit{Нм}.
\end{multlined}
\end{equation}

В результате получили следующие исходные данные для расчета требуемых моментов и мощностей на валу двигателей:\par


	\begin{tabular}{ll}
\( \ddot \alpha _{0}=34,9\textit{рад/с}^{2} \)				& \( \ddot \beta _{0}=17.45 \textit{рад/с}^{2} \)  \\
\( I_{y}=0.327\textit{кгм}^{2} \)						&  \( I_{z}=0.0084 \textit{кгм}^{2} \) \\
\( m_{1}=15.7\textit{кг} \)								& \( m_{2}=1.131 \textit{кг} \) \\
\( \dot \alpha ^{max}=10.47\textit{рад/с} \)				& \( \dot \beta ^{max}=5.235\textit{рад/с} \) \\
\( M_{\textit{\textit{нагр1}}}=2.1425\textit{Нм} \)		& \( M_{\textit{\textit{нагр2}}}=0.250755\textit{Нм} \)
	\end{tabular}


В силу (\labelcref{eq:p4:1}) получим требования к двигателям по азимуту и углу места:\par

- к пусковому моменту на валу ротора

\begin{equation}%\tag{44}
\label{eq:p4:4+3}
\begin{multlined}
M_{\textit{дв1}} \geq 2 \left[ 2.1425 + \left( 0.327 \cdot 34.9 \right)  \right] =27.1096\textit{Нм} \rightarrow 30\textit{Нм}, \\
M_{\textit{дв2}} \geq 2 \left[ 0.25 + 0.0084 \cdot 17.45 \right] =0.79316\textit{Нм} \rightarrow 1\textit{Нм};
\end{multlined}
\end{equation}

- к потребляемой мощности

\begin{equation}%\tag{44}
\label{eq:p4:4+4}
\begin{multlined}
P_{1}=27.11 \cdot 10.47=283.84\textit{Вт} \rightarrow 300\textit{Вт}, \\
P_{2}=0.8 \cdot 5.235=4.2\textit{Вт} \rightarrow  \left( 5 \div 10 \right) \textit{Вт}.
\end{multlined}
\end{equation}
С учетом полученных требований можно использовать наиболее приемлемые моментные двигатели (МД) типа ДБМ (согласованные с Заказчиком) для управления ОЭП:

\begin{itemize}
	\item \textit{по азимуту}: 5\ ДБМ 120-5-2-3 (с запасом по пусковому моменту, новая разработка с 2007г.)  и 5 ДБМ 120-2-1-3 (без запаса по пусковому моменту, новая разработка с 2007г.), ДБМ 150-4-0,6-3 (без запаса по пусковому моменту, серийный с 2005г.), ДБМ 150-4-1,5-3 (с запасом по пусковому моменту, серийный с 2005г.)
	\item \textit{по углу места} – 3 ДБМ 70-1,1-1,3-3 (с запасом по пусковому моменту, серийный с 2005г.)
\end{itemize}

Характеристики указанных моментных двигателей приведены в таблице \ref{tab:DBM}. \par

\begin{landscape}


%%%%%%%%%%%%%%%%%%%% Table No: 10 starts here %%%%%%%%%%%%%%%%%%%%


\begin{table}[!h]
	\caption{Характеристики приводов ДБМ}%
	\label{tab:DBM}% label всегда желательно идти после caption
	\begin{longtable}{|m{10cm}|m{2.2cm}|m{2.2cm}|m{2.2cm}|m{2.2cm}|m{2.2cm}|m{2.2cm}|}
		\hline
		%row no:1
		Название & 
		5ДБМ120- 5-2-3 & 
		{3ДБМ120- \par 1-0.8-3} & 
		{5ДБМ120- \par 2-1-3} & 
		{3ДБМ150- \par 4-0.6-3} & 
		{ДБМ150- \par 4-1.5-3} & 
		{3ДБМ70- \par 1.1-1.3-3} \\
		\hline
		%row no:2
		Наружный диаметр статора, мм & 
		120 & 
		{120} & 
		{120} & 
		{150} & 
		{150} & 
		{70} \\
		\hline
		%row no:3
		Внутренний диаметр статора & 
		47 & 
		{} & 
		{} & 
		{72} & 
		{72} & 
		{28} \\
		\hline
		%row no:4
		Число пар полюсов & 
		8 & 
		{8} & 
		{8} & 
		{8} & 
		{8} & 
		{8} \\
		\hline
		%row no:5
		Число фаз & 
		3 & 
		{3} & 
		{3} & 
		{3} & 
		{3} & 
		{3} \\
		\hline
		%row no:6
		Номинальное напряжение питания, В & 
		27 & 
		{18} & 
		{18} & 
		{27} & 
		{27} & 
		{27} \\
		\hline
		%row no:7
		Частота вращения холостого хода, об/мин & 
		2200-2500 & 
		{740-900} & 
		{1080} & 
		{560-700} & 
		{1720-1910} & 
		{1120-1420} \\
		\hline
		%row no:8
		Пусковой момент, Н$\ast$ м, не менее & 
		54 & 
		{6.5} & 
		{29.4} & 
		{26} & 
		{37.4} & 
		{5.5} \\
		\hline
		%row no:9
		Сопротивление фазы постоянному току, Ом & 
		0.029- 0.035 & 
		{0.63- 0.77} & 
		{0.123} & 
		{0.22- 0.27} & 
		{0.1- 0.13} & 
		{0.45- 0.52} \\
		\hline
		%row no:10
		Электромагнитная постоянная времени фазы, мс, не более & 
		 & 
		{} & 
		{} & 
		{1.2} & 
		{1.8} & 
		{0.4} \\
		\hline
		%row no:11
		Приведенные к фазе коэффициенты момента Cm, Н$\ast$ м/А ; ЭДС Ce, В$\ast$ с/рад & 
		0.0075-0.0085 & 
		{0.17-0.21} & 
		{11-0.14} & 
		{0.23-0.28} & 
		{0.09-0.1} & 
		{0.1-0.14} \\
		\hline
		%row no:12
		Момент инерции ротора, кг$\ast$ м\textsuperscript{2} & 
		0.75 10\textsuperscript{-5} & 
		{1 10\textsuperscript{-5}} & 
		{0.26 10\textsuperscript{-5}} & 
		{3 10\textsuperscript{-3}} & 
		{3 10\textsuperscript{-3}} & 
		{2.5 10\textsuperscript{-4 }} \\
		\hline
		%row no:13
		Момент сопротивления, Н$\ast$ м, не более & 
		0.6 & 
		{0.1} & 
		{0.16} & 
		{0.4} & 
		{0.4} & 
		{0.11} \\
		\hline
		%row no:14
		Предельный ток, А & 
		 & 
		{} & 
		{} & 
		{66} & 
		{165} & 
		{40} \\
		\hline
		%row no:15
		Электромеханическая постоянная времени, мс & 
		 & 
		{} & 
		{} & 
		{10} & 
		{14.3} & 
		{7} \\
		\hline
		%row no:16
		Масса, кг & 
		 & 
		{} & 
		{} & 
		{3} & 
		{3} & 
		{1.2} \\
		\hline
		%row no:17
		Материал магнитов & 
		Nd-FE-B & 
		{EN38S} & 
		{EN38S} & 
		{Nd-Fe-B} & 
		{Nd-Fe-B} & 
		{Nd-Fe-B} \\
		\hline
		
\end{longtable}
\end{table}
%%%%%%%%%%%%%%%%%%%% Table No: 10 ends here %%%%%%%%%%%%%%%%%%%%
\end{landscape}

\begin{figure}[ht]
	\centering
	\includegraphics[width=0.8\linewidth]{image37} 
	\caption{Внешний вид 5ДБМ120- 5-2-3}
	\label{fig:5DBM120}
\end{figure}

\begin{figure}[ht]
	\centering
	\includegraphics[]{image38} 
	\caption{Внешний вид 3ДБМ120-1-0.8-3, 5ДБМ120-2-1-3}
	\label{fig:3DBM120}
\end{figure}

\begin{figure}[ht]
	\centering
	\includegraphics[width=0.8\linewidth]{image39} 
	\caption{Внешний вид 3ДБМ150-4-0.6-3, ДБМ150-4-1.5-3, 3ДБМ70-1.1-1.3-3}
	\label{fig:3DBM70}
\end{figure}

Для дальнейших исследований выбраны наиболее предпочтительные МД, удовлетворяющие требованиям ТЗ:\par

\begin{itemize}
	\item \textit{по азимуту} –  5ДБМ 120-5-2-3,
	\item \textit{по углу места} – 3 ДБМ 70-1,1-1,3-3.
\end{itemize}




\section{Уравнения движения изолированных каналов} \label{ch:ch4/sect3}

Используя линеаризованные уравнения (\ref{eq:p3:59}) полученные в разделе \ref{ch:ch3/sect10} составим зависимости положений приводов от управляющего сигнала, возмущающего воздействия и движений ЛА. В виду незначительного влияния перекрестных связей запишем уравнения (\ref{eq:p3:59}) в следующем виде:

\begin{equation}
\label{eq:p4:s3.1}
\begin{multlined}
\left( a_{11}p^{2}+b_{11}p \right)  \alpha + 
a_{11} p^2  \psi 
=
k_{1}  u_{1}- 
 M_{\textit{тр.1}},\\
\left( a_{22}p^{2}+b_{22}p \right)  \beta + c_{22}+
d_{23} p^2  \vartheta
=\\
k_{2}  u_{2} - M_{\textit{тр.2}},
\end{multlined}
\end{equation}

Перепишем уравнения относительно обобщенных координат:

\begin{equation}
\label{eq:p4:s3.2}
\begin{multlined}
\alpha= 
\frac{k_{1}}{a_{11}p^{2}+b_{11}p} u_{1} - 
\frac{1}{a_{11}p^{2}+b_{11}p} M_{\textit{тр.1}} - 
\frac{a_{11} p^2}{a_{11}p^{2}+b_{11}p}  \psi  ,\\
\beta=
\frac{k_{2}}{a_{22}p^{2}+b_{22}p} u_{2} - 
\frac{1}{a_{22}p^{2}+b_{22}p} M_{\textit{тр.2}} - \\
\frac{d_{23} p^2}{a_{22}p^{2}+b_{22}p} \vartheta -
\frac{1}{a_{22}p^{2}+b_{22}p} c_{22},
\end{multlined}
\end{equation}

Из выражения (\ref{eq:p4:s3.2}) следует, что уголы поворота ОЭП зависит не только от управления, но и от моментов трения, дисбаланса ($c_{22}=M_{\textit{дб}}$) и движений носителя. Для дальнейшего использования выражение (\ref{eq:p4:s3.2}) запишем в виде:

\begin{equation}
\label{eq:p4:s3.3}
\begin{aligned}
\alpha= 
W_{n1}(p) u_{1} - 
W_{f1}(p) M_{\textit{тр.1}} - 
W_{\psi 1}(p) \dot \psi  ,\\
\beta=
W_{n2}(p) u_{2} - 
W_{f2}(p) (M_{\textit{дб}} + M_{\textit{тр.2}}) - 
W_{\vartheta}(p) \dot \vartheta,
\end{aligned}
\end{equation}

где
ПФ азимута по управлению:
\begin{equation}
\label{eq:p4:s3.4}
\begin{multlined}
W_{n1}(p) = 
\frac{k_{1}}{a_{11}p^{2}+b_{11}p} = 
\frac{K_{n1}}{(T_{M1} T_{e1} p^2 + T_{M1} p +1)p} = 
\frac{K_{n1}}{R_1(p) p};
\end{multlined}
\end{equation}
ПФ азимута по возмущению:
\begin{equation}
\label{eq:p4:s3.5}
\begin{multlined}
W_{f1}(p) = \frac{1}{a_{11}p^{2}+b_{11}p} =  
\frac{K_{m1} (T_{e1} p +1)}{(T_{M1} T_{e1} p^2 + T_{M1} p +1)p} = 
\frac{K_{m1} R_{e1}(p)}{R_1(p)p};
\end{multlined}
\end{equation}
ПФ азимута от движения носителя:
\begin{equation}
\label{eq:p4:s3.6}
\begin{multlined}
W_{\psi}(p) = \frac{a_{11} p^2}{a_{11}p^{2}+b_{11}p} = 
\frac{T_{M1}(T_{e1} p +1)}{(T_{M1} T_{e1} p^2 + T_{M1} p +1)} = 
\frac{T_{M1}R_{e1}(p)}{R_1(p)};
\end{multlined}
\end{equation}
где 
\begin{equation}
\label{eq:p4:tt1}
\begin{aligned}
T_{M1} =\frac{R_1 (B_1 + B(\beta_0))}{C_{m1}C_{e1}},
T_{e1} = L_1/R_1,
\end{aligned}
\end{equation}

\begin{equation}
\label{eq:p4:kk1}
\begin{aligned}
K_{n1}=\frac{1}{C_{e1}},
K_{m1}= \frac{R_1}{C_{m1}C_{e1}}.
\end{aligned}
\end{equation}

ПФ угла места по управлению:
\begin{equation}
\label{eq:p4:s3.7}
\begin{multlined}
W_{n2}(p) = \frac{k_{2}}{a_{22}p^{2}+b_{22}p} = 
\frac{K_{n2}}{(T_{M2} T_{e2} p^2 + T_{M2} p +1)p} = 
\frac{K_{n2}}{R_2(p)p};
\end{multlined}
\end{equation}
ПФ угла места по возмущению:
\begin{equation}
\label{eq:p4:s3.8}
\begin{multlined}
W_{f2}(p) = \frac{1}{a_{22}p^{2}+b_{22}p} =  
\frac{K_{m2}(T_{e2} p + 1)}{(T_{M2} T_{e2} p^2 + T_{M2} p +1)p} = 
\frac{K_{m2}R_{e2}(p)}{R_2(p)p};
\end{multlined}
\end{equation}
ПФ угла места от движения носителя:
\begin{equation}
\label{eq:p4:s3.9}
\begin{multlined}
W_{\vartheta}(p) = \frac{d_{23} p^2}{a_{22}p^{2}+b_{22}p} = 
\frac{d_{23} T_{M2}(T_{e2} p + 1)}{(T_{M2} T_{e2} p^2 + T_{M2} p +1)} = 
\frac{d_{23} T_{M2}R_{e2}(p)}{R_2(p)};
\end{multlined}
\end{equation}
где 
\begin{equation}
\label{eq:p4:tt2}
\begin{aligned}
T_{M2} =\frac{R_1 C_2}{C_{m2}C_{e2}},
T_{e2} = L_2/R_2,
\end{aligned}
\end{equation}

\begin{equation}
\label{eq:p4:kk2}
\begin{aligned}
K_{n2}=\frac{2}{C_{e2}},
K_{m2}= \frac{R_2}{C_{m2}C_{e2}}.
\end{aligned}
\end{equation}

Запишем в матричной форме

\begin{equation}
\label{eq:p4:s3.3+}
\begin{aligned}
q = W_n(p) u - W_f(p) M_{\textit{н}} - W_{\textit{ла}} (p)\omega,
\end{aligned}
\end{equation}
где
\begin{equation}
\label{eq:p4:s3.4+}
\begin{aligned}
W_n(p) = K_n R(p)^{-1} \dfrac{1}{p}= \left( \begin{array}{c c}
W_{n1}(p) & 0 \\
0 & W_{n2}(p)
\end{array}\right),
u = \left( \begin{array}{c}
u_1 \\
u_2
\end{array}\right),\\
W_f(p) = K_m R_e(p) R(p)^{-1} \dfrac{1}{p}= \left( \begin{array}{c c}
W_{f1}(p) & 0 \\
0 & W_{f2}(p)
\end{array}\right),
M_{\textit{н}} = \left( \begin{array}{c}
M_{\textit{тр.1}} \\
M_{\textit{дб}} + M_{\textit{тр.2}}
\end{array}\right),\\
W_{\textit{ла}}(p) = T_m R_e(p) R(p)^{-1} = \left( \begin{array}{c c}
W_{\psi}(p) & 0 \\
0 & W_{\vartheta}(p)
\end{array}\right),
\omega = \left( \begin{array}{c}
\dot \psi \\
\dot \vartheta
\end{array}\right),           
\end{aligned}
\end{equation}
здесь
\begin{equation}
\label{eq:p4:s3.10}
\begin{aligned}
K_n = \left( \begin{array}{c c}
K_{n1} & 0 \\
0 & K_{n2}
\end{array}\right),
K_m = \left( \begin{array}{c c}
K_{m1} & 0 \\
0 & K_{m2}
\end{array}\right), 
T_m = \left( \begin{array}{c c}
T_{m1} & 0 \\
0 & T_{m2}
\end{array}\right), \\
R(p) = \left( \begin{array}{c c}
T_{M1} T_{e1} p^2 + T_{M1} p +1 & 0 \\
0 & T_{M2} T_{e2} p^2 + T_{M2} p +1
\end{array}\right), \\
R_e(p) = \left( \begin{array}{c c}
T_{e1} p + 1 & 0 \\
0 & T_{e2} p + 1
\end{array}\right).
\end{aligned}
\end{equation}

\begin{comment}
\subsection{Уравнения движения привода по азимуту} \label{subsec:ch4/sect3/sub1}

Линеаризованные уравнения движения азимутального привода совместно с объектом управления запишутся [29] [30] [31] [32] [33] [34]:

\subsection{Уравнения движения привода по углу места} \label{subsec:ch4/sect3/sub2}
\end{comment}

\section{Синтез алгоритмов} \label{ch:ch4/synthesis}

\subsection{Синтез алгоритмов и моделирование программного устройства} \label{ch:ch4/sect2+}

\subsubsection{Режим наведения} \label{subsec:ch4/sect2/sub1}

Режим наведения должен осуществляться в пределах заданых углов относительно ЛА ($\varDelta\alpha$) за время ($\varDelta t$). Для решения этой задачи управления рассмотрим оптимальное управление по быстродействию [3, 27].

Задача устройства состоит в том чтобы выдать оптимальную траекторию движения основываясь на следующих условиях:
\begin{enumerate}
	\item средняя скорость разворота должна удовлетворять требованиям ТУ: 
	$\frac{\varDelta\alpha}{\varDelta t}$
	\item скорость движения в начальный и конечным момент должна быть равна 0
	\item программный угол не должен выходить за заданые пределы
\end{enumerate}

Построим программное устройство (ПУ), предназначенное для выдачи требуемой программной траектории движения оптической оси на основе решения уравнений:
\begin{equation}
\label{eq:p4:2+.1}
\begin{alignedat}{2}
\varDelta\alpha = \dfrac{\ddot{\alpha}{\varDelta t}^2}{4}
\end{alignedat}
\end{equation}

Схема моделирования ПУ приведена на рисунке \ref{fig:model_control}. Фрагмент решения уравнений (\ref{eq:p4:2+.1}) приведен далее (рисунок \ref{fig:model_control_graph}).

\begin{figure}[ht]
	\centering
	\includegraphics[width=1.0\linewidth]{model_control} 
	\caption{Схема моделирования программного устройства}
	\label{fig:model_control}
\end{figure}

Для задания необходимой программной траектории достаточно задать угол (обозначен как "target" на рисуноке \ref{fig:model_control}), на который  необходимо перевести оптическую ось ОЭП, задать скорость разворота (обозначен как "dx" и "dt" на рисуноке \ref{fig:model_control}) и задать предельные углы разворота (обозначен как "max" и "min" на рисуноке \ref{fig:model_control}). 

\begin{figure}
	\centering
	\includegraphics[width=0.7\linewidth]{model_control_graph}
	\caption{График ускорения, скорости и координаты программного движения}
	\label{fig:model_control_graph}
\end{figure}

\subsection{Синтез возмущений} \label{ch:ch4/sect2+1}

Из технических требований (таблица \ref{tab:SAU_PARAM}) известны условия при которых должна работать ОЭС. Эти воздействия можно описать в виде уравнений:
\begin{itemize}
	\item \( \omega  \left( t \right) = 12 \textit{град/с} \) ,
	\item \(  \omega _{1} ( t ) =1.38sin \left( 6.28 t \right) \),
	\item \( \omega_2 (t) = 0.21 sin(t) \).
\end{itemize}

\begin{figure}[ht]
	\centering
	\includegraphics[width=1.0\linewidth]{1} 
	\caption{Блок модели внешних возмущений}
	\label{fig:la_model}
\end{figure}

При этом учтены следующие предельно-возможные возмущения: 
\begin{itemize}
	\item момент силы нагрузки $M_{\textit{н1}} = 2.1425 \textit{Нм}$,
	\item скорость \( \dot \psi  \left( t \right) = 12 \textit{град/с} \) ,
	\item ускорения \(  \ddot \psi _{1} \left( t \right) =1.38sin \left( 6.28 t \right) \),
	\item \( \ddot \psi_2 (t) = 0.21 sin(t) \).
\end{itemize}

\subsection{Синтез алгоритмов управления ОЭП} \label{ch:ch4/sect4-}

В соответствии с ТЗ управление ОЭП проводится в двух режимах:
наведения ОЭП на объект наблюдения (ОН) и стабилизации оптической оси относительно направления на ОН. 

Синтез алгоритмов управления проводился частотным методом \cite[]{Bessekerski} для каждого из режимов. На основе полученных законов управления разработаны компьютерные модели (КМ) линейной и нелинейной САУ в среде Simulink MatLAB и проведены исследования динамики САУ с помощью КМ в режимах наведения и стабилизации при действии возмущений, оговоренных в ТЗ.

Структурная схема САУ изолированного канала показана на рисунке \ref{fig:structured_SAU}.

\begin{figure}[ht]
	\centering
	\includegraphics[width=0.8\linewidth]{structured_SAU}
	\caption{Структурная схема САУ}
	\label{fig:structured_SAU}
\end{figure}

Программное устройство определяет режим работы и задает целевое значение положения ротора канала, регулятор обеспечивает оптимальную работу привода, привод - объект управления , 
датчик угла измеряет положение ротора, "носитель" определяет внешние возмущающие факторы влияющие на систему.

%\subsection{Азимут} \label{ch:ch4/sect4-/sub1}

В соответствии с уравнением (\ref{eq:p4:s3.3+}) и параметрами (таблица \ref{tab:DBM}) приводов выбраных ранее в разделе \ref{ch:ch4/sect2} МД 5ДБМ 120-5-2-3 (новая разработка) для азимута и МД ДБМ 70-1,1-1,3-3 для угла места оценим параметры приводов необходимые для дальнейших расчетов при исходных данных для:
\begin{itemize}
	\item \textit{канала азимута (1)}
	
	\[  R_1 = 0.32 \textit{Ом}, 
	T_{e1} = 0.1 \textit{мс}, 
	J_{p1} = 0.75 \cdot 10^{-5} \textit{кг м}^2,  
	u_\textit{н1} = 27 \textit{В},
	i_\textit{н1} = 75 \textit{А} \]
	
	\begin{equation}
	\label{eq:p4:sec4/2}
	\begin{multlined}
	C_{e1}=
	\frac{u_\textit{н1}-i_{\textit{н1}}R_{1}}{ \omega _{\textit{н}}}=
	\frac{27-75 \cdot 0.32}{11.1}=
	0.27 \textit{Вс/рад},
	\end{multlined}
	\end{equation}
	
	\begin{equation}
	\label{eq:p4:sec4/3}
	\begin{multlined}
	C_{m1}=
	\frac{M_{\textit{н}}}{i_{\textit{н}}}=
	\frac{27.11}{75}=0.36 \textit{Нм/А},
	\end{multlined}
	\end{equation}
	
	\item \textit{канала угла места (2)}
	
	\[  R_2 = 0.485 \textit{Ом}, 
	T_{e2} = 0.4 \textit{мс}, 
	J_{p2} = 0.2510 \cdot 10^{-5} \textit{кг м}^2,  
	u_\textit{н2} = 27 \textit{В},\]
	
	\[C_{e2} = 0.12 \textit{Вс/рад}, 
	C_{m2} = 0.12 \textit{Нм/А},
	\]
	
	\item \textit{внешних возмущениях} (таблица \ref{tab:SAU_PARAM})
	
	\begin{equation}
	\label{eq:p4:sec4/1}
	\begin{multlined}
	\omega _{\textit{н}}=
	\alpha _{\textit{пр}} + A_{1} \omega _{1} + A_{2} \omega _{2} + \alpha _{0} =\\
	10.47 + 0.21 \cdot 1 + 0.035 \cdot 6.28 + 0.21 = 11.1 \textit{рад/с},
	\end{multlined}
	\end{equation}

\end{itemize}
где  \(  \omega _{\textit{н}}- \) максимальное значение номинальной угловой скорости прибора, А\textsubscript{1} , А\textsubscript{2 }- амплитуды колебаний на частотах 0.16 Гц и 1Гц,  \( \dot \alpha _{0} \)  - угловая скорость носителя,  \( \dot \alpha _{\textit{пр}} \) - программный поворот прибора.

\begin{itemize}
	\item \textit{канала азимута (1)}
	\begin{equation}
	\label{eq:p4:sec4/4}
	\begin{alignedat}{2}
	T_{M1}=
	\frac{ \left( J_{\textit{н1}}+J_{01} \right) R_{1}}{C_{m1}C_{e1}}=\frac{ \left( 0.327 + 0.75 \cdot 10^{-5} \right) 0.32}{0.36 \cdot 0.27}=
	1.08 \textit{c} ,\\
	K_{n1} = \frac{1}{C_{e1}} = \frac{1}{0.27} = 3.7 \textit{рад/Вс},\\
	K_{m1} = \frac{R_1}{C_{e1} C_{m1}} = \frac{0.32}{0.27 \cdot 0.36} = 3.29 \textit{рад/Нмс},
	\end{alignedat}
	\end{equation}
	
	\item \textit{канала угла места (2)}
	\begin{equation}
	\label{eq:p4:sec4/4+}
	\begin{alignedat}{2}
	T_{M2}=
	\frac{ \left( J_{\textit{н2}}+J_{02} \right) R_{2}}{C_{m2}C_{e2}}=\frac{ \left( 0.2510 \cdot 10^{-5} + 8.4 \cdot 10^{-3} \right) 0.485}{0.12 \cdot 0.12}=
	0.2913 \textit{c} ,\\
	K_{n2} = \frac{1}{C_{e2}} = \frac{1}{0.12} = 8.33 \textit{рад/Вс},\\
	K_{m2} = \frac{R_2}{C_{e2} C_{m2}} = \frac{0.485}{0.12 \cdot 0.12} = 33.68 \textit{рад/Нмс},
	\end{alignedat}
	\end{equation}
	
\end{itemize}



Для оценки требуемой добротности САУ по скорости запишем передаточную функцию по ошибке в соответствии с уравнениями:\par


\begin{equation}%\tag{414}
\label{eq:p4:414}
\begin{alignedat}{2}
\begin{comment}
\left. \begin{array}{ll}
\alpha =W_{n1} \left( p \right) U_{1}-W_{f1} \left( p \right) M_{H1}-W_{ \psi } \left( p \right) \dot \psi\\
\Delta  \alpha = \alpha _{\textit{пр}}- \alpha\\
U_{1}=K_{\textit{д1}}\cdot K_{y1} \cdot W_{k1} \left( p \right)  \Delta  \alpha \\
\end{array}  \right\rbrace \\ 
\end{comment}
\left. \begin{array}{ll}
q = W_n(p) u - W_f(p) M_{\textit{н}} - W_{\textit{ла}} (p)\omega\\
\Delta  q = q_{\textit{пр}}- q\\
U = K_{\textit{д}}\cdot K_{y} \cdot W_{k} \left( p \right)  \Delta  q \\
\end{array}  \right\rbrace  
\end{alignedat}
\end{equation}
где
$ \Delta  q = \left( \begin{array}{l}
\Delta \alpha\\
\Delta \beta
\end{array} \right)  $ - рассогласование,

$ q_{\textit{пр}} = \left( \begin{array}{l}
\alpha_{\textit{пр}}\\
\beta_{\textit{пр}}
\end{array} \right)  $ - программное управление,

$K_{\textit{д}} = \left( \begin{array}{cc}
K_{\textit{д1}} &0\\
0&K_{\textit{д2}}
\end{array} \right)  $ - коэффициенты усиления датчиков,

$K_{\textit{y}} = \left( \begin{array}{cc}
 K_{y1}  &0\\
0& K_{y2} 
\end{array} \right)  $ - коэффициенты усиления преобразователей,

$W_{k} \left( p \right) = \left( \begin{array}{cc}
 W_{k1} \left( p \right) &0\\
0& W_{k2} \left( p \right)
\end{array} \right)  $ - передаточные функции регуляторов.
Из уравнений (\ref{eq:p4:414}) получим:

\begin{comment}
\begin{equation}%\tag{415}
\label{eq:p4:415}
\begin{alignedat}{2}
\Delta  \alpha =\frac{1}{1+W_{1} \left( p \right) } \alpha _{\textit{пр}}-\frac{W_{f1} \left( p \right) }{1+W_{1} \left( p \right) }M_{H1}-\frac{W_{ \psi } \left( p \right) }{1+W_{1} \left( p \right) } \dot \psi,
\end{alignedat}
\end{equation}
\end{comment}

\begin{equation}%\tag{415}
\label{eq:p4:415}
\begin{alignedat}{2}
\Delta  q = 
(W(p)+E_2)^{-1} q _{\textit{пр}}-
(W(p)+E_2)^{-1} W_{f1} \left( p \right) M_{H}-\\
(W(p)+E_2)^{-1} W_{\textit{ла}} (p) \omega,
\end{alignedat}
\end{equation}

где 
\begin{comment}
\begin{equation}
\label{eq:p4:415+}
\begin{alignedat}{2}
 W_{1} \left( p \right) =
 K_{\textit{д1}}K_{\textit{у1}}W_{k1} \left( p \right) W_{n1} \left( p \right) =
 W_{k1} \left( p \right) \frac{K_{c1}}{ \left( T_{M1}T_{e1}p^{2}+T_{M1}p+1 \right) p}; \\
  K_{c1}=K_{\textit{д1}}K_{\textit{у1}}K_{k1}K_{n1}.
\end{alignedat}
\end{equation}
\end{comment}
\begin{equation}
\label{eq:p4:415+}
\begin{alignedat}{2}
W \left( p \right) =
K_{\textit{д}}K_{\textit{у}}W_{k} \left( p \right) W_{n} \left( p \right) =
W_{k} \left( p \right) K_{c} R(p)^{-1}  \dfrac{1}{p}; \\
K_{c}=K_{\textit{д}}K_{\textit{у}}K_{k}K_{n}.
\end{alignedat}
\end{equation}

В установившемся режиме ($t\rightarrow0, p\rightarrow0$) имеем\par
\begin{comment}
\begin{equation}
\label{eq:p4:415+2}
\begin{alignedat}{2}
\Delta  \alpha =
\frac{1}{K_{c1}} \dot \alpha_{\textit{пр}} + \frac{K_{m1}}{K_{c1}}M_{H1} - \frac{T_{M1}}{K_{c1}} \dot \psi ,
M_{Н1}=M_{\textit{ТР1}}+M_{\textit{дб1}}=2.1425\textit{Нм}
\end{alignedat}
\end{equation}
\end{comment}
\begin{equation}
\label{eq:p4:415+2}
\begin{alignedat}{2}
\Delta  q =
K_{c}^{-1} \dot q_{\textit{пр}} + K_{m} K_{c}^{-1} M_{\textit{н}} - T_{m} K_{c}^{-1} \omega ,\\
M_{\textit{н}} = \left( \begin{array}{c}
M_{\textit{тр.1}} \\
M_{\textit{дб}} + M_{\textit{тр.2}}
\end{array}\right)=
\left( \begin{array}{c}
2.1425\textit{Нм} \\
0.25 \textit{Нм}
\end{array}\right)
\end{alignedat}
\end{equation}

Из решения уравнения (\ref{eq:p4:2+.1}) следует, что в конечной координате наведения (при \textit{t} = 0.6c)  \( \dot q _{\textit{пр}}=0 \) , тогда коэффициент усиления разомкнутой системы можно определить по формуле \par
\begin{comment}
\begin{equation}
\label{eq:p4:416}
\begin{alignedat}{2}
K_{c1} \geq \frac{K_{m1}M_{H1}+T_{M1} \ddot \psi ^{max}}{ \Delta  \alpha ^{\textit{доп}}}
\end{alignedat}
\end{equation}
\end{comment}

\begin{equation}
\label{eq:p4:416}
\begin{alignedat}{2}
K_{c} \geq E_2 {(K_{m}M_{\textit{н}}+T_{m} \dot \omega ^{max})}{ \Delta  {q ^{\textit{доп}}}^{-1}}
\end{alignedat}
\end{equation}

Оценим  \( K_{c}, \dot \omega ^{max} \), где 

$A_1 = 12 \textit{град} = 0.21 \textit{рад}, \omega_1 = 1 \frac{1}{c}$

$A_2 = 2 \textit{град} = 0.035 \textit{рад}, \omega_2 = 6.28 \frac{1}{c}$

$\varDelta q^{\textit{доп}} =  0.5 \textit{град} = 0.0087 \textit{рад}$
\[ 
K_{c}=
\left( \begin{array}{cc}
\frac{3.29 \cdot 2.14 + 1.08 \cdot 1.6}{0.0087} \frac{1}{c}&0 \\
0& \frac{33.68 \cdot 0.25 + 0.2913 \cdot 1.6}{0.0087}
\end{array}\right)\frac{1}{c}=
\left( \begin{array}{cc}
940.85 &0 \\
0& 1021.38
\end{array}\right) \frac{1}{c} \rightarrow
\]

\[
\left( \begin{array}{cc}
1200 &0 \\
0& 1250
\end{array}\right)\frac{1}{c}
. 
\] \par

Оценим устойчивость системы (\ref{eq:p4:414}), используя частотный критерий Найквиста \cite[]{Bessekerski}. Для этого, используя полученные параметры системы, построим частотные характеристики (ЧХ): (ЛAX –логарифмическая амплитудная характеристика, ЛФХ – логарифмическая фазовая характеристика) разомкнутой системы $W_1(j \omega)$ (рисунки \ref{fig:LogAmpChar1},\ref{fig:LogAmpChar2}).\par


\begin{figure}[ht]
	\centering
	\includegraphics[width=1.0\linewidth]{image51} 
	\caption{LAX и LФХ разомкнутой системы. 1-ЛАХ, 2-ЛФХ ЧХ1($W_{k1}(p)=1 $), 3-ЛАХ, 4-ЛФХ ЧХ2 ($T_{k11} = 1.08 c$, $T_{k21} = 0.0001 c$}
	\label{fig:LogAmpChar1}
\end{figure}

\begin{figure}[ht]
	\centering
	\includegraphics[width=1.0\linewidth]{image66} 
	\caption{LAX и LФХ разомкнутой системы. 1-ЛАХ, 2-ЛФХ ЧХ1($W_{k1}(p)=1 $), 3-ЛАХ, 4-ЛФХ ЧХ2 ($T_{k12} = 0.04 c$, $T_{k22} = 0.001 c$}
	\label{fig:LogAmpChar2}
\end{figure}

Из анализа ЧХ1 (рисунки \ref{fig:LogAmpChar1},\ref{fig:LogAmpChar2}, ЧХ-1) видно, что исходная САУ (\ref{eq:p4:414}) без коррекции (${W_k}_i=1$) находится на границе устойчивости. Для устойчивости рассматриваемой системы с требуемыми запасами (по фазе  $ \geq $  (45-60) град., L~$\geq$ (6-15) дб),\ необходимые для обеспечения приемлемых переходных процессов САУ, введем корректирующее последовательное фазоопережающее устройство с ПФ:

\begin{equation}%\tag{417}
\label{eq:p4:417}
\begin{alignedat}{2}
{W_{k}}_{ii} \left( p \right) =K_{ki}\frac{ \left( T_{k1i}p+1 \right) }{ \left( T_{k2i}p+1 \right) }.
\end{alignedat}
\end{equation}

Выбором параметров (\ref{eq:p4:417}) можно обеспечить устойчивость и требуемые запaсы устойчивости и качество переходного процесса. 
\begin{itemize}
	\item \textit{Для азиимута}\par
	$T_{k11} = 1.08 c$, $T_{k2} = (0.0001  \div 0.001) c$  САУ устойчива (рисунок \ref{fig:LogAmpChar1} ЧХ-2), имея запасы устойчивости по амплитуде $\varDelta L = (24 \div 19)$ дБ и фазе $\varDelta \varphi = (86 \div 53)$ град,
	\item \textit{Для угла места}\par
	$T_{k12} = 0.04 c$, $T_{k22} =  0.001 c$  САУ устойчива (рисунок \ref{fig:LogAmpChar2} ЧХ-2), имея запасы устойчивости по амплитуде $\varDelta L = 30$ дБ и фазе $\varDelta \varphi = 72$ град.
\end{itemize}

На основе полученных параметров синтезированного регулятора, модели ПУ (Рисунок \ref{fig:model_control}) и структурой схемы (рисунок \ref{fig:structured_SAU}) разработана компьютерная имитационная модель (КИМ) линейной САУ (рисунок \ref{fig:az}) в соответствии с уравнениями (\ref{eq:p4:414}) , используя MatLAB Simulink. Содержание блока «Программное движение» показано на рисунке \ref{fig:model_control}, «Возмущения». 

\begin{figure}[ht]
	\centering
	\includegraphics[width=1.0\linewidth]{az} 
	\caption{КИМ1,2 линейной САУ в режиме наведения}
	\label{fig:az}
\end{figure}

Исходные данные для моделирования азимутального канала указаны в листинге \ref{lst:az}, для канала угла места в листинге \ref{lst:um}. 

\begingroup
\captiondelim{ } % разделитель идентификатора с номером от наименования
\lstinputlisting[caption={Исходные данные для КИМ1 (канал азимута)},label={lst:az}]{listings/az_3.m}
\endgroup

\begingroup
\captiondelim{ } % разделитель идентификатора с номером от наименования
\lstinputlisting[caption={Исходные данные для КИМ2 (канал угла места)},label={lst:um}]{listings/um_3.m}
\endgroup

Разработанная КИМ1,2 САУ для режима наведения (рисунок \ref{fig:az}) позволяет проводить исследования динамики процесса управления ОЭП, варьируя параметрами регулятора, объекта управления и возмущающими воздействиями в широких пределах при различных заданных углах наведения в диапазоне  \( \alpha = \pm 180 \textit{град}\), \( \beta = (+30 \div -60) \textit{град}\).

На рисунке ниже (рисунках \ref{fig:az_},\ref{fig:um_}) приведены переходные и установившиеся процессы наведения ОЭП на угол равный \( \alpha = 180  \textit{град}, \beta = -60  \textit{град} \) при действии указанных выше возмущений ($M_{\textit{н}}, \omega(t)$). Из анализа процессов наведения (рисунки~\ref{fig:az_}, \ref{fig:um_}) при максимальном угле наведения (\( \alpha = 180  \textit{град}, \beta = -60  \textit{град} \)) погрешность наведения не превышает $\Delta \alpha = 0.5 \textit{град}, \Delta \beta = 0.5 \textit{град}$, при этом коэффициенты датчика и усилителя должны быть не менее $K_{\textit{д1}} K_{\textit{у1}} K_{\textit{к1}} = 320 \textit{В/рад}, K_{\textit{д2}} K_{\textit{у2}} K_{\textit{к2}} = 150 \textit{В/рад}$.

\begin{figure}[ht]
	\centering
	\includegraphics[width=1.0\linewidth]{az_} 
	\caption{Процессы наведения ОЭП по каналу азимута}
	\label{fig:az_}
\end{figure}

\begin{figure}[ht]
	\centering
	\includegraphics[width=1.0\linewidth]{um_} 
	\caption{Процессы наведения ОЭП по каналу угла места}
	\label{fig:um_}
\end{figure}

\subsection{Синтез нелинейной САУ} \label{subsec:ch4/sect4/sub2}

Основными нелинейностями, влияющие на динамику САУ, являются:
\begin{itemize}
	\item насыщение в усилителе мощности (\textit{U\textsubscript{max} =  27В});
	\item дискретность преобразователя сигналов  СКВТ в качестве датчика угла (ДУ), для 14 разрядов дискрета составляет ($\varDelta_{\textit{ду}} = 1.3183 \textit{угл. мин} = 3.83 \cdot 10^{-4} \textit{рад}$), принципиальная схема датчика показана на рисуноке \ref{fig:SKVT2}, в соответствии с \cite[]{1310HM025} для 14 разрядного преобразования датчик можно представить в виде дискретного звена со следующей ПФ:
	\begin{equation}
	\label{eq:p4:SKVT}
	\begin{alignedat}{2}
	G ( z ) =\frac{ 1 - a z^{-1} }{ 1 - b z^{-1} } = \frac{ 1 - 8191/8192 z^{-1} }{ 1 - 8182/8192 z^{-1} },
	\end{alignedat}
	\end{equation}
	имеет ЧХ разомкнутого контура показанную на рисунке \ref{fig:SKVT3};
	\item частота широтно-импульсной модуляции (ШИМ), описывается следующей передаточной функцией:
	\begin{equation}
	\label{eq:p4:PWM}
	\begin{alignedat}{2}
	W ( p ) =k_{pwm} \frac{ e^{-p \tau_0 } }{ T_{pwm} p + 1 } = k_{pwm} \frac{ e^{-p \tau_0 } }{ T_{pwm} p + 1 },
	\end{alignedat}
	\end{equation}
	при частоте счетчика $f_y~=~16000$Гц
\end{itemize} 

\begin{figure}[ht]
	\centering
	\includegraphics[width=0.8\linewidth]{SKVT2} 
	\caption{принципиальная схема цифровой части датчика СКВТ \cite[]{1310HM025}}
	\label{fig:SKVT2}
\end{figure}

Для обработки сигналов СКВТ используется микросхема преобразователя сигналов датчиков перемещения
1310НМ025 со следующими характеристиками:
\begin{itemize}
	\item Напряжение цифрового и аналоговогопитания 3,0 – 5,5 В;
	\item Разрядность преобразователякоординаты настраиваетсяпользователем от 8 до 16 бит;
	\item Частота возбуждения датчиков от 0 до 20 кГц;
	\item Генератор опорного сигнала счастотой от 20 Гц до 20 кГц;
	\item Возможно одновременноеподключение двух датчиков СКВТ или ЛРДТ;

\end{itemize}

\begin{figure}[ht]
	\centering
	\includegraphics[width=1.0\linewidth]{SKVT3} 
	\caption{Вид типичной АЧХ и ФЧХ контура (LBW=14) \cite[]{1310HM025}}
	\label{fig:SKVT3}
\end{figure}

САУ по углу места, как и САУ по азимуту, является нелинейной и импульсной. В связи с этим будем решать те же вопросы, связанные с изучением влияния нелинейностей и дискретного характера сигналов на динамические свойства САУ. \par

\subsection{Синтез цифровой САУ } \label{subsec:ch4/sect4/sub3}

(ЦСАУ) наиболее просто проводить в частотной области методами, разработанными для непрерывных систем \cite{Bessekerski},\cite{Karpov29}. Синтез параметров будем проводить на основе  критерия  устойчивости Найквиста и частотных критериев качества регулирования   ($M_i < 1.25$, запасы устойчивости: по фазе $\varDelta \varphi_i \geq 45 $ град. и по амплитуде $\varDelta L_i \geq 6$ дБ).\par

В связи с этим отметим некоторые условия, которые необходимо выполнить и которые ограничивают область синтезируемых параметров регулятора линейной системы.\par

При исследовании устойчивости и качества регулирования САУ совместно с цифровыми вычислительными устройствами (ЦВУ) в контуре управления необходимо построить желаемую логарифмическую частотную характеристику (ЛАХ) с учетом следующих условий \cite{Bessekerski},\cite{Karpov29}.\par

\begin{enumerate}
	\item Для обеспечения требований по устойчивости и запасу устойчивости в соответствии с критерием Найквиста необходимо дополнительно выполнить неравенство:\par
	\begin{equation}%\tag{418}
	\label{eq:p4:418}
	\omega_{\textit{ср}}< \omega _{\textit{гр}}=\frac{2}{T}
	\end{equation}
	где  \(  \omega _{\textit{ср}}\) - частота среза непрерывной системы,  \(  \omega _{\textit{гр}}\) - граничная частота,  \( T- \) период повторения (квантование по времени) ЦВУ;	
	\item Малые постоянные времени ($T_i <  \omega_{\textit{ср}}^{-1}$) должны удовлетворять условию: $T_i < 0.5 T$;
	\item Переход оси нуля децибел асимптотической ЛАХ непрерывной части, разомкнутой САУ проходит при отрицательном наклоне 20 \textit{дб/дек};
	\item При построении ЛАХ разомкнутой САУ с ЦВУ в высокочастотной области  (\( \omega > \omega_{\textit{ср}} \)) следует учитывать сумму малых постоянных времени ( \(  \sum T_{i} \) ), при этом необходимо выполнить  ещё одно дополнительное условие\par	
	\begin{equation}%\tag{419}
	\label{eq:p4:419}
	\frac{T}{2}+ \sum T_{i} \leq \frac{1}{ \omega _{\textit{ср}}}\frac{M}{ \left( M+1 \right)},
	\end{equation}
	которое позволяет применять частотные методы расчета линейных систем для исследования САУ с ЦВУ в высокочастотной области.\par
\end{enumerate}

Таким образом, при выполнении условий  ЛАХ разомкнутой САУ с ЦВУ в контуре управления в низкочастотной области  (\(  \omega< \omega _{\textit{ср}} \))  можно заменить ЛАХ линейной разомкнутой САУ в этой же области. Так как в обоих каналах управления используются усилитель и датик одной конструкции, то расчеты периода квантования аналогичны для обоих каналов.


Оценим периоды квантования: 
\begin{itemize}
	\item Для ШИМ усилителя $T_{pwm} = 1/16000 = 0.0000625 c$;
	\item Для ДУ период квантования оценим из условия обеспечения измерения угла отработки (подробнее в разделе \ref{subsec:ch4/sect4/sub2}) двигателем минимальной угловой скорости
	\begin{equation}
	\alpha_0 = 12 \textit{град/сек}, T_{\textit{ду}} = \frac{\varDelta_{\textit{ду}}}{\dot \alpha_0}=
	\frac{0.0219}{12}=0.00183 c;
	\end{equation}	
\end{itemize}

Из расчетов следует, что предельным периодом квантования ЦСАУ будет $T = 0.00183 c$. С учетом условий (\ref{eq:p4:418}),(\ref{eq:p4:419}) при $M=1.1$ оценим:\par

\begin{equation}
\label{eq:p4:x}
\begin{multlined}
\omega_{\textit{гр}}=\frac{2}{T}=\frac{2}{0.00183}=1092.9 c^{-1},\\ 
\sum T_{i}=T_{\textit{e1}}+T_{k2}=0.0001+0.00005=0.00015 c, \\
\omega _{\textit{ср}} \leq 491.84 c^{-1} \rightarrow 250 c^{-1} \rightarrow log(250) = 2.398
\end{multlined}
\end{equation}

С учетом условий (\ref{eq:p4:418}),(\ref{eq:p4:419}) получены требования к линейной части ЦСАУ.\par

На основе проведенных расчетов синтезируем частотным методом \cite{Bessekerski} корректирующее звено (\ref{eq:p4:417}) с новыми параметрами: \par

\begin{equation}
T_{k11} = 0.05 c, T_{k21} = 0.00005 c, T_{k12} = 0.05 c, T_{k22} = 0.00005 c, 
\end{equation}\par

обеспечивающие  устойчивость ЦСАУ с запасами устойчивости
\begin{itemize}
	\item \textit{для канала азимута}
	
	по амплитуде ($\varDelta L = 50$ дБ) и фазе  ($\varDelta \varphi = 83$ град) рисунок \ref{fig:DSAU},
	\item \textit{для канала угла места}
	
	по амплитуде ($\varDelta L = 28$ дБ) и фазе ($\varDelta \varphi = 82$ град)  рисунок \ref{fig:DSAU2}.
\end{itemize}

Для обеспечения требуемой точности наведения добротность САУ по скорости определена в пределах 
\begin{itemize}
	\item \textit{для канала азимута}
	
	К\textsubscript{с1}= (3590 – 3700) с\textsuperscript{-1}, соответственно - К\textsubscript{д1}К\textsubscript{к1 }К\textsubscript{у1}= (700-1000) В/рад,
	\item \textit{для канала угла места}
	
	К\textsubscript{с2}= 1250с\textsuperscript{-1}, соответственно - К\textsubscript{д2}К\textsubscript{у2} К\textsubscript{к2}= 150 В/рад.
\end{itemize}

\begin{figure}[!ht]
	\centering
	\includegraphics[width=1.0\linewidth]{image55} 
	\caption{ЛАХ и ЛФХ разомкнутой линейной части ЦСАУ канала азимута}
	\label{fig:DSAU}
\end{figure}

\begin{figure}[!ht]
	\centering
	\includegraphics[width=1.0\linewidth]{image69} 
	\caption{ЛАХ и ЛФХ разомкнутой линейной части ЦСАУ канала угла места}
	\label{fig:DSAU2}
\end{figure}

\section{Моделирование и исследование динамики} \label{subsec:ch4/modeling}

\subsection{Моделирование и исследование динамики ЦСАУ} \label{subsec:ch4/sect4/sub4}

На основе синтезированных регуляторов САУ и с учетом указанных нелинейностей разработаны КИМ ЦСАУ (рисунок \ref{fig:digital_az}). Модели с учетом нелинейностей КИМ позволяют изменять величины дискретности датчика, частоту ШИМ усилителя и насыщения в УМ и проводить исследования динамики нелинейной САУ при заданных пределах этих параметров. 


\begin{figure}
	\centering
	\includegraphics[width=1.0\linewidth]{digital_az} 
	\caption{КИМ 3,4 нелинейной ЦСАУ в режиме наведения}
	\label{fig:digital_az}
\end{figure}

Исходные данные для моделирования азимутального канала указаны в листинге \ref{lst:digital_az}, для канала угла места в листинге \ref{lst:digital_um}. 

\begingroup
\captiondelim{ } % разделитель идентификатора с номером от наименования
\lstinputlisting[caption={Исходные данные для КИМ3 (канал азимута)},label={lst:digital_az}]{listings/az_d.m}
\endgroup

\begingroup
\captiondelim{ } % разделитель идентификатора с номером от наименования
\lstinputlisting[caption={Исходные данные для КИМ4 (канал угла места)},label={lst:digital_um}]{listings/um_d.m}
\endgroup

На рисунках \ref{fig:az_digital} – \ref{fig:az_digital6} приведены результаты моделирования динамики наведения ОЭП с учетом указанных нелинейностей в канале азимута на угол  $\alpha=180^0$, а на рисунках \ref{fig:um_digital} – \ref{fig:um_digital5} в канале угла места на угол  $\beta=60^0$. Из анализа процессов наведения следует, что:\par

\textbf{Азимут}

- при совместном учете насыщения УМ, дискретности ДУ и ШИМ УМ, погрешность наведения на угол $180^0$  составляет $ \varDelta\alpha < 0.5^0$ (рисунок \ref{fig:az_digital}).\par

\begin{figure}
	\centering
	\includegraphics[width=1.0\linewidth]{az_digital1} 
	\caption{КИМ 3, Процессы наведения ОЭП с учетом насыщения УМ, дискретности ДУ и ШИМ УМ ($K_{\textit{д1}} K_{\textit{у1}} K_{\textit{к1}} = 700 \textit{В/рад}$)}
	\label{fig:az_digital}
\end{figure}

- изменение напряжения на выходе УМ в процессе наведения приведены на рисунке ниже (рисунок \ref{fig:az_digital2}). Период переключений напряжений: ($0.12   \div   1.25$) мс.\par

\begin{figure}
	\centering
	\includegraphics[width=1.0\linewidth]{az_digital2} 
	\caption{Изменение напряжения на выходе усилителя мощности в процессе наведения}
	\label{fig:az_digital2}
\end{figure}

- учет только насыщения УМ ($U_{max}=27\textit{В}$)  приводит к увеличению времени переходного процесса, погрешность наведения достигает величины  $ \varDelta\alpha < 1.2^0$  (рисунок \ref{fig:az_digital3}). При   $K_{\textit{д1}} K_{\textit{у1}} K_{\textit{к1}} = 1000 \textit{В/рад}$  погрешность наведения можно уменьшить до величины $\varDelta\alpha < 0.45^0$ . \par

\begin{figure}
	\centering
	\includegraphics[width=1.0\linewidth]{az_digital3_saturation} 
	\caption{Процессы наведения ОЭП с учетом насыщения УМ}
	\label{fig:az_digital3}
\end{figure}

- при учете только ШИМ УМ ($f_g = 16000$ Гц) и насыщения УМ погрешность наведения существенно не изменилась ($\varDelta\alpha < 1.2^0$) (рисунок \ref{fig:az_digital4});\par

\begin{figure}
	\centering
	\includegraphics[width=1.0\linewidth]{az_digital4_pwm} 
	\caption{Процессы наведения ОЭП с учетом ШИМ УМ}
	\label{fig:az_digital4}
\end{figure}

- учет только дискретности датчика угла ($\varDelta_{\textit{ду}} = 1.3183 \textit{угл. мин} = 3.83 \cdot 10^{-4} \textit{рад}$) приводит к появлению пульсаций ($\varDelta\alpha_{\textit{пул}} < 0.08^0$) прибора из-за дискретности датчика угла, при этом погрешность наведения равна $\varDelta\alpha < 1.2^0$ (рисунок \ref{fig:az_digital5}).\par

\begin{figure}
	\centering
	\includegraphics[width=1.0\linewidth]{az_digital5_sensor} 
	\caption{Процессы наведения ОЭП с учетом дискретности ДУ (14 разрядов)}
	\label{fig:az_digital5}
\end{figure}
- уменьшение времени в ПУ до 0.5 секунд приводит

\begin{figure}
	\centering
	\includegraphics[width=1.0\linewidth]{az_digital6_05sec} 
	\caption{Процессы наведения ОЭП с корректировкой ПУ}
	\label{fig:az_digital6}
\end{figure}









\textbf{Угол места}


-при совместном учете насыщения УМ, дискретности ДУ и ШИМ УМ, погрешность наведения на угол 180\textsuperscript{0}\  равна\   = 0,2\textsuperscript{0} (рисунок \ref{fig:um_digital}).\par

\begin{figure}[ht]
	\centering
	\includegraphics[width=1.0\linewidth]{um_digital1} 
	\caption{КИМ 3, Процессы наведения ОЭП с учетом насыщения УМ, дискретности ДУ и ШИМ УМ (Кд1Ку1 Кк1=150 В/рад)}
	\label{fig:um_digital}
\end{figure}

-При этом изменение напряжения на выходе УМ в процессе наведения приведены на рисунке ниже (Рисунок \ref{fig:um_digital2}). Период переключений напряжений: (0,12 $  \div $  1,25) мс.\par

\begin{figure}[ht]
	\centering
	\includegraphics[width=1.0\linewidth]{um_digital2} 
	\caption{Изменение напряжения на выходе усилителя мощности в процессе наведения}
	\label{fig:um_digital2}
\end{figure}

- учет только насыщения УМ (\textit{U\textsubscript{max} =  27В})\  приводит к увеличению времени переходного процесса, погрешность наведения достигает величины  =0,65\textsuperscript{0}\  (Рис.13).\ При\ \   К\textsubscript{д1}К\textsubscript{у1} К\textsubscript{к1}=1000\ В/рад  погрешность наведения можно уменьшить до величины  =0,46\textsuperscript{0} . \par


\begin{figure}[ht]
	\centering
	\includegraphics[width=1.0\linewidth]{um_digital3_saturation} 
	\caption{Процессы наведения ОЭП с учетом насыщения УМ}
	\label{fig:um_digital3}
\end{figure}
сунок 4.13. Рис.13. Процессы наведения ОЭП с учетом насыщения УМ


- при учете только ШИМ УМ (\textit{f\textsubscript{у }}= 8000Гц) и насыщения УМ погрешность наведения существенно не изменилась ( =0.65\textsuperscript{0} ) (Рисунок 4.14);\par

\begin{figure}[ht]
	\centering
	\includegraphics[width=1.0\linewidth]{um_digital4_pwm} 
	\caption{Процессы наведения ОЭП с учетом ШИМ УМ}
	\label{fig:um_digital4}
\end{figure}
сунок 4.14. Рис.14. Процессы наведения ОЭП с учетом ШИМ УМ




- учет только дискретности датчика угла (\textsubscript{ду}= 5,2724 угл. мин ) приводит к появлению пульсаций (\textsubscript{пул} = 0,08\textsuperscript{0 }) прибора из-за дискретности датчика угла, при этом погрешность наведения равна =0,38\textsuperscript{0} (Рисунок 4.15).\par

\begin{figure}[ht]
	\centering
	\includegraphics[width=1.0\linewidth]{um_digital5_sensor} 
	\caption{Процессы наведения ОЭП с учетом дискретности ДУ (14 разрядов)}
	\label{fig:um_digital5}
\end{figure}
сунок 4.15. Рис.15. Процессы наведения ОЭП с учетом дискретности ДУ 12 разряда

\subsection{Моделирование и исследование динамики изолированных каналов управления ОЭП в режиме стабилизации} \label{ch:ch4/sect6}

Режим стабилизации ОЭП выполняется после режима наведения на заданную координату пространства. В связи с этим рассмотрим возможность максимального использования регуляторов, синтезированных ранее для режима наведения, в режиме стабилизации. В этом режиме при действии возмущений, идущих от носителя, а также моментов трения и дисбаланса  объекта управления, оптическая ось прибора должна следить за объектом наблюдения с точностью =30угл.мин  в течении 10с. Предполагаем, что в режиме стабилизации носитель может совершать эволюции, представленные в таблице ниже (Таблица 4.3), 

\subsubsection{Моделирование и исследование динамики нелинейной САУ по курсу} \label{subsec:ch4/sect6/sub1}


\subsubsection{Моделирование и исследование динамики нелинейной САУ по углу места} \label{subsec:ch4/sect6/sub2}


\section{Анализ результатов исследований и определение требований к элементам САУ} \label{ch:ch4/sect7}


\section{Выводы по главе} \label{ch:ch4/sect8}



Некоторый текст.

\clearpage